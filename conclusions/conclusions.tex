%!TEX root = ../thesis.tex
\chapter{Conclusions and future work}
\label{chap:conclusions}

In this work, I have created, tested, and validated a new \acrfull{is} that recovers the \glspl{pdf} of star clusters kinematic and photometric observables. In particular, the proper motions and apparent magnitudes. When the distance to the cluster is known, the \glspl{pdf} of these observables can be transformed into transverse velocity and luminosity distributions. In addition, using the cluster age and a mass-luminosity relation the latter can be transformed into a mass distribution. A compilation of kinematic and mass distributions of \glspl{nyc} in diverse environments and spanning the ages of the early phases of cluster evolution will allow the astrophysical community to understand the effects that the environments and the internal dynamics have in the origin and evolution of stellar clusters. Since the majority of the stars form in clusters, understanding how these form and evolve will shed light into the knowledge of the past and future of our galaxy. 

The \gls{is} presented in this work solves most of the issues found in current methodologies of cluster members determination and analysis. Furthermore, it steps ahead of the current paradigm of these methodologies, which is the classification of objects in cluster and field populations, and moves into a new one: \textit{learning the kinematic and photometric distributions of the cluster populations}, both single and binary stars.

Thanks to its Bayesian framework, the Bayesian Hierarchical Models methodology and the use of weakly informative priors specifically, this new \gls{is} solves the issues present in current classifiers from the literature and minimises their biases. In particular, it fulfils its objective by properly propagating the observable multivariate and heteroscedastic uncertainties, using the valuable information provided by objects with missing values in their observables, and eliminating the sampling bias associated with the use of only high membership probability objects.

To learn the posterior distributions of the parameters in the cluster model, the \gls{bhm}, as we call this new \gls{is}, uses a combination of learning techniques, the \acrlong{pso} of \citet{Kennedy1995,Clerc2002} and the \emph{emcee} \acrlong{mcmc} method of \citet{Foreman2013}. Both are implemented in \acrlong{hpc} environments to use large data sets ($10^5$ to $10^6$ objects) and multidimensional observable spaces. While the multidimensionality of the observable space (proper motions and photometry) add valuable information to constrain the cluster model and render it accurate, the large size of the data set increases its precision. 

The performance of the \gls{bhm} as a classifiers was analysed on synthetic data sets. The results of this analysis show that at an optimal probability threshold of $p=0.84$ the it has a \acrlong{tpr} (recovery rate) of $90.0\pm0.05$\% and a contamination rate of $4.3\pm0.2$\%. The \acrlong{auc} \acrlong{roc} is 0.99 which indicates that it as an excellent classifier. The synthetic analysis reveals that the expected value of the contamination in the kinematic and photometric cluster distributions is $5.8\pm 0.2$\%.

The \gls{bhm} has been thoroughly tested and validated on the \acrlong{ddr2} \citep{Bouy2015}. Since the Pleiades is one of the most studied clusters in the history of astronomy, it offers the perfect case study for the benchmarking of an \gls{is} aiming at the analysis of \gls{nyc}. In addition, the high precision and carefully designed observations of the \gls{dance} project for the Pleiades cluster provide the necessary detailed information and sufficient statistics to benchmark the \gls{bhm} and constrain its posterior distribution, in the high dimensionality (85) of its parametric space specifically. 

The results of applying the \gls{bhm} to the \acrfull{rdr2} data set yield the following astrophysical results.

\begin{itemize}
\item The by-product cluster membership probabilities, which are now delivered as full \glspl{pdf}, show that (see Section \ref{sect:memberscomparison}):
\begin{itemize}
\item  There is an outstanding agreement of 99\% with the classification process done by \citet{Bouy2015} on the one hundred thousands objects data set.
\item From the 1967 pleiads candidate members found in this work, 205 are new ones. It represents 10\% of the cluster population. 
\item From the candidate members of \citet{Rebull2016}, 91\% are classified as members. This figure is better than our estimated value of the \acrlong{tpr}, $ 90.0 \pm0.05$.
\end{itemize}
\item The model selection analysis of the \acrfull{psd} performed on the \acrfull{hmps} of candidate members recovered by the \gls{bhm} shows that (see Section \ref{sect:PSDresults}):
\begin{itemize}
\item The King's profile \citep{King1962} is not just a good model due to its physical interpretability, but also has the largest Bayesian evidence when compared to classical profiles from the literature.
\item There is enough Bayesian evidence to support luminosity segregation of the Pleiades cluster.
\item The Pleiades data set supports, with large Bayesian evidence, \gls{psd} models with biaxial symmetry indicating the non negligible ellipticity of the cluster ($\epsilon=0.13$).
\end{itemize}
\item The kinematic and photometric distribution of the cluster populations show that (see Section \ref{sect:PMresults} and \ref{sect:luminosity}):
\begin{itemize}
\item The \acrfull{emb} fraction is $9.9\pm0.1$\%.
\item The centroids of the cluster proper motions are $\{16.30,-39.68\}\,\rm{mas\cdot yr^{-1}}$ and $\{15.71,-40.31\}\,\rm{mas\cdot yr^{-1}}$ for single and \gls{emb}, respectively. 
\item The derived luminosity distributions in the infrared bands ($J,H, \rm{and} \,K_s$) in the completeness interval are in good agreement with the previous ones of \citet{Bouy2015}.
\end{itemize}
\item The derived \acrfull{pdsmd} \cite[using an age of 120 \gls{myr} and the BT-Settl isochrone model of][]{Allard2012} shows (see Sections \ref{sect:massdistributionresults} and \ref{sect:massontime}):
\begin{itemize}
\item A general agreement with the \glspl{imf} of \citet{Chabrier2005} and \citet{Thies2007}, in the intermediate mass range specifically.
\item The \glspl{imf} of \citet{Chabrier2005} and \citet{Thies2007} predict too many low-mass stars and \acrfull{bd}.
\item A trend of depletion in low-mass stars and \gls{bd} with cluster age, when comparing it with the Trapezium and Hyades \gls{pdsmd} from the literature.
\end{itemize}
\end{itemize}

By minimising bias, propagating uncertainties, including correlations and doing a comprehensive modelling of the data particularities (missing values and heteroscedastic uncertainties) the \gls{bhm}, developed, tested and validated in the present work, delivers the kinematic and photometric distributions of the cluster single and equal-mass binary populations. This new intelligent system represents the state of the art in the statistical analysis of \glspl{nyc}.

\section{Current issues and future improvements}

Despite the fact that the \gls{bhm} presented here solves the issues present in current methodologies from the literature, the following is a non-exhaustive list of important aspects to tackle in the near future.

\begin{itemize}
\item The \gls{bhm} must be able to deal with:
\begin{itemize}
\item Unresolved binaries of different mass ratios. Although the \gls{csn} has already been included in the \gls{bhm} to deal with these objects, its application still demands fine tuning of the \gls{hpc} elements (see Section \ref{subsect:cluster}).
\item Proper motion and photometry from other surveys (e.g. \emph{Pan-STARRS}, \emph{Gaia}, \emph{LSST}). Although so far untested, the \gls{bhm}, thanks to its ability to cope with large amounts of missing values, it can deal with data sets from heterogeneous origins. For example, \emph{Gaia} proper motion data (together with its correlated uncertainties) plus \emph{Pan-STARRS} photometry, can be directly plugged into the \gls{bhm} to improve the uncertainties of the kinematic and photometric cluster distributions. 
\item Other observables. As shown with the projected spatial distribution, other observables like radial velocities from the \emph{Gaia-ESO} survey and parallaxes from \emph{Gaia}, will give useful information to further refine the lists of candidate members.
\item Extinction. The treatment of interstellar extinction is of paramount importance and is mandatory for the application of the \gls{bhm} in younger and embedded clusters. Once done, it will allow the unravelling of kinematic and photometric distributions of embedded young clusters.

\item Unresolved multiple systems. Although it is expected that these systems represent a small fraction of the cluster population \cite[only 4\% for the triple systems][]{Duquennoy1991}, they nevertheless have an important contribution to the mass distribution. 

\item Clusters with superimposed populations. The treatment of superimposed populations is also important to minimise contamination.

\item The white dwarfs population. Although their numbers are negligible on \gls{nyc}, their mass contribution nevertheless must be considered to properly constrain the \gls{pdsmd} and the \glspl{imf}. 
\end{itemize}
\item The time needed to learn the \gls{bhm}. Currently, the \gls{bhm} takes four weeks to run in a 80 \glspl{cpu} computing cluster. Although the \gls{dance} team has lately translated the \gls{cpu} code into \gls{gpu} code, the latter still needs improvement, in memory allocation and parallel distribution particularly.
\item The choosing of prior distributions in other non-well studied \glspl{nyc}. In \glspl{nyc} where the prior information is missing, other kind of prior distributions must be used (e.g. objective or non-informative priors). Nevertheless, the \gls{bhm} still provides the less subjective path to parametric inference.
\item Choosing the appropriate number of \gls{gmm} components. As with the previous point, where the prior information is not available, we can nevertheless use non-informative priors and large number of \gls{gmm} components, so that those non-required components had small weight and covariance matrices. A model selection analysis must be implemented.
\item Uncertainties to the mass-luminosity relation. Although non related to the \gls{bhm}, the mass-luminosity relation is used univocally to transform a luminosity into mass. However, this knowledge is uncertain and its uncertainty must be incorporated into the derived mass distributions.
\end{itemize}

Once these issues will be solved, the improved \gls{bhm} will render the inventory of cluster kinematic and luminosity distributions that will allow the \gls{dance} team to fulfil the objective of understanding the effects that the environment (i.e. initial conditions) and the internal dynamics (e.g. evaporation, ejection, mass segregation) have on the clusters formation and evolution.

Finally, I  list some future perspectives of the \gls{bhm}.

\begin{itemize}
\item It will eventually become a free and open-source code for the use of the astronomical community.
\item Although it has been created for the analysis of star clusters, it can also be applied to stellar associations.
\item The empirical colour-magnitude relations that it learns can be used to constrain the theoretical evolutionary models once corrected by individual distances and extinctions.
\item The lists of candidate members and equal-mass binaries that it delivers can be used to perform efficient searches and follow-ups.
\item In the \emph{Gaia} era, it will continue to provide not just accurate lists of candidate members, but also unbiased kinematic and photometric distributions with correctly propagated uncertainties for objects all along from the high mass domain down to the planetary mass regime.
\end{itemize}

USE OF OTHER COLOUR INDICES
VARYING INTRINSIC DISPERSION

spatial incompleteness due to bright sources

Photometric incompleteness in the i band. Panstarss






