%!TEX root = ../thesis.tex
\chapter{Introduction}
\label{chap:introduction}

The majority of the stars forms in stellar clusters. \citet{2000AJ....120.3139C} reports that $50-70\%$ of very young, $\leq10$ millions of years (Myr), and $25-70\%$ of the young, $\leq100$ Myr, stellar populations are formed in clusters. \citet{2003AJ....126.1916P} and \citet{2003ARA&A..41...57L} find that among 80\% to 90\% of the stars are formed in clusters with more than 100 members. Furthermore, as indicated by the former authors, these populous clusters represent 22\% of the regions where stars form. The remaining of the star forming regions are small associations, with 5 to 30 members, where only up to 10\% of the stars are formed. However, only less than 7\% of these populous clusters survives as gravitationally bounded clusters when reaching an age of a few hundred Myr \citep{2003ARA&A..41...57L}. The remaining 93\% of the star forming regions will become unbounded and their stars will freely populate the galaxy. Thus, to understand the general rules that govern how the majority of stars forms, as well as the properties of the stars that populate our galaxy, it is crucial to fully decode the formation and early evolution of stellar clusters. 

Astrophysicists, like archeologists and palaeontologists, can not willingly reproduce the vast majority of their studied phenomena. Although some experiments can be performed in specific situations (e.g. the chemical and physical  properties of dust and gas), astrophysics remains an observational science. For this reason, to test the validity of their hypotheses, astrophysicists rely on statistical studies carried out over carefully designed observations. In particular, the understanding of the star formation process requires carefully designed observations of  stellar clusters whose ages cover the early stages of their evolution.

The objective of this work is the construction, test and validation of an statistical tool, an intelligent system specifically, that given the carefully designed data of a stellar cluster, recovers the statistical distributions of its populations. In particular, it should deliver the cluster luminosity distribution, which can be transformed into the mass distribution given an evolutionary model and the cluster age. The mass distribution is the fundamental product of the star formation process. It contains the fingerprints of the early phases of star formation and subsequent cluster evolution.

An homogeneous and precise mass distribution inventory for clusters of diverse ages and forming environments will allow the astrophysical community to test the current theories of the star formation process. In particular, it will allow to solve the questions about universality of the initial mass function (IMF, see next Section) and the role play by the physical properties of the cluster environment.    

The remaining of this Chapter is structured as follows. In Section \ref{sect:IMF}, I describe the importance of the initial mass distribution and some of its current models. In Section \ref{sect:numerical_simulations}, I report the current status of numerical simulations of cluster formation and its impact on the understanding of the star formation process. In Section \ref{sect:DANCeproject}, I describe the project DANCe, its objectives and its carefully designed observations of stellar clusters. This work is part of that project and makes use of its observations. In Section \ref{sect:current_methodologies}, I comment on the past and current methodologies to select members of clusters and associations. Finally in Section \ref{sect:newIS}, I briefly describe the methodology adopted for this new statistical tool, and its advantages over the previous works. 

\section{The initial mass function of stellar clusters}
\label{sect:IMF}

In his seminal work \citep{Salpeter1955}, Edwin Salpeter defined the \emph{original mass function}, $\xi(M)$, as

\begin{equation}
\rm{d}N= \xi(M)\rm{d}(\log_{10} M) \frac{\rm{d}t}{T_0},\nonumber
\end{equation}
where $\rm{d}N$ is the number of stars in the mass range $\rm{d}M$ created in the time interval $\rm{d}t$ per cubic parsec, and $T_0$ is the age of the galaxy. Following \citet{Chabrier2003b},  the mass function (MF) at the observed time $t$, is

\begin{equation}
\xi(\log_{10} M)=\frac{\rm{d}N}{\log_{10} M},\nonumber
\end{equation}

where $N$ is the stellar number density, and M the mass. The Initial Mass Function (IMF) is defined as the MF at the time of stellar formation $t=t_0$. The logarithmic transformation of the mass,
\begin{equation}
\xi(M)=\frac{1}{M \ln 10} \xi (\log_{10} M),\nonumber
\end{equation}
is convenient due to the large range of masses covered by the star formation process.

Notice that neither the IMF nor the MF are probability density functions (PDFs) of the mass (see Section \ref{sect:introprobability} for the definition of a PDF). Nevertheless, they can be transformed into PDFs by a normalisation constant, which can be computed by integrating them as functions of the mass over the mass domain. In this work, I will use the PDF of the mass (PDM) in the logarithmic scale, denoted by $\xi_L (\log_{10}M)$, as a proxy for the MF. Thus,

\begin{equation}
\xi_L (\log_{10} M) \propto \xi (\log_{10} M).\nonumber
\end{equation}

The measuring and understanding of the IMF is a central topic in the study of star formation. It is also essential in other areas of astrophysics, from planetary formation, where it appears that the mass of the host star plays an important role in the formation of the planetary system \cite[see for example][]{2015ApJ...814..130M}, to galactic evolution \citep{1998ASPC..142....1K} and cosmology \cite[see for example][]{2012MNRAS.423.3601N}. 

The theories that predict the origin of the IMF can be categorised into deterministic and stochastic \citep{Offner2014}. The former postulate that stellar masses are deterministically inherited from the initial core masses via accretion from the gas reservoir of the parent molecular cloud. Thus, the IMF can be directly mapped from the distribution of initial core masses, and the understanding of the former reduces to that of the latter. On the other hand, stochastic models postulate that the stellar masses are independent of the initial core masses. Among these models, there are those proposing that stellar masses are determined by dynamical interactions and competitive accretion. For more details see \citet{Offner2014} and references therein. 

  The observational studies of the IMF are conditioned on the ages of the stellar populations under analysis (their MF at their corresponding ages), and rely deeply on the assumed processes that link the observed present-day MF to the IMF. While the resulting models for the IMF are analytical functions of the mass, the observed MF are commonly expressed with points, histograms or kernel density estimations\footnote{KDEs are non-parametric ways to estimate a probability density function by means of an independently and identically distributed sample drawn from it.} (KDEs).

The most common IMFs are the power-law functions of Salpeter \citep{Salpeter1955}, Miller and Scalo \citep{1979ApJS...41..513M}, and Kroupa \citep{2001MNRAS.322..231K,2002Sci...295...82K,2013pss5.book..115K,Thies2007,2008MNRAS.390.1200T}, and the log-normal functions of Chabrier \citep{Chabrier2003a,Chabrier2003b,Chabrier2005}. Other functional forms include the truncated exponential \citep{2001AGM....18S0551D} and the Pareto-Levy family distribution \citep{2012MNRAS.423.1018C}. In behalf of simplicity, I will only explain the classical ones of Salpeter, Chabrier, and Kroupa.

\citet{Salpeter1955} derived his famous IMF using a luminosity function resulting from the compilation of the works of \citet{1939POMin...7....1L,1941NYASA..42..201L} and \citet{1925PGro...38D...1V,1936PGro...47....1V}. Then, he transformed it into a MF using a mass-luminosity relation that he obtained after adopting a series of masses and luminosities from the literature. His MF has the form

\begin{equation}
\xi(M)=0.03 \left(\frac{M}{M_{\odot}}\right)^{-1.35},\nonumber
\end{equation}
with $M$ in the range $0.3\,M_{\odot}$ to  $17\,M_{\odot}$. Its units are $ M_{\odot}^{-1} \cdot pc^{-3}$.

\citet{Chabrier2003a,Chabrier2003b} derived his Present-day MF (PDMF) from the nearby luminosity functions of both the $V$ band \citep{1986AJ.....91..621D} and the $K$ band \citep{1990ApJ...350..334H} for the solar neighbourhood ($\leq$5 pc). He used the \citet{2000A&A...364..217D} and \citet{1998A&A...337..403B} mass-magnitude relations in $V$ and $K$ bands, respectively, to transform the luminosity functions into masses. Then, he fitted a log-normal form to the single objects with masses below 1 $M_{\odot}$. The PDMF he found (in units of $(\log M_{\odot})^{-1}\cdot pc^{-3}$) is,

\begin{equation}
\xi(\log m)_{m\leq1M_{\odot}}=0.158_{-0.046}^{+0.051} \times \exp{\left\{-\frac{(\log m - \log 0.079_{-0.016}^{+0.021})^2}{2 \times (0.69_{-0.01}^{+0.05})^2}\right\}}.\nonumber
\end{equation}

As shown by \citet{1986FCPh...11....1S}, 1 $M_{\odot}$ is the limit at which the PDMF starts to differ from the IMF. Therefore, \citet{Chabrier2003b} uses his PDMF as the IMF below the 1 $M_{\odot}$ limit. Above it, he adopts the Salpeter IMF,

\begin{equation}
\xi(\log m)_{m>1M_{\odot}}= 4.43\times10^{-2}\cdot m^{-1.3\pm0.3}.\nonumber
\end{equation}

As shown by \citet{1991MNRAS.251..293K}, the discrepancies between the luminosity functions derived from photographic samples and from trigonometric parallaxes of nearby stars can be accounted with unresolved binaries. For this reason, \citet{Chabrier2003a} derived the Present-Day system mass function (PDSMF) for unresolved systems. It takes into account each object as a possible unresolved binary or system. This PDSMF has the form,

\begin{equation}
\xi(\log m)_{m\leq1M_{\odot}}=0.086\times \exp{\left\{-\frac{(\log m - \log 0.22)^2}{2 \times (0.57)^2}\right\}},\nonumber
\end{equation}
with the same normalisation and coefficients of the IMF above 1 $M_{\odot}$.

Later, \citet{Chabrier2005} included in his analysis the revision that \citet{2002AJ....124.2721R} made to the sample of \citep{1986AJ.....91..621D}, and the extended sample of 8 pc from \citet{2004AJ....128..463R}. His new PDMF is

 \begin{align}
\xi(\log m)_{m\leq1M_{\odot}}=&0.093\times \exp{\left\{-\frac{(\log m - \log 0.2)^2}{2 \times (0.55)^2}\right\}},\nonumber\\
\xi(\log m)_{m>1M_{\odot}}=& 0.041\times m^{-1.3\pm0.3}.\nonumber
\end{align}

And his new PDSMF is
 \begin{align}
\xi(\log m)_{m\leq1M_{\odot}}=&0.076\times \exp{\left\{-\frac{(\log m - \log 0.25)^2}{2 \times (0.55)^2}\right\}},\nonumber\\
\xi(\log m)_{m>1M_{\odot}}=& 0.041\times m^{-1.3\pm0.3}.
\end{align}

The canonical IMF of Kroupa \citep{2013pss5.book..115K} is a three-segment power-law, with two segments describing the IMF of the stars, while the third does it for Brown-Dwarfs (BD)\footnote{Brown-dwarfs are substellar objects with masses in the range from 10 to 80 Jupiter masses (0.01$M_{\odot}$ to 0.075 $M_{\odot}$). Because there is no fusion of hydrogen in their cores, these objects are not classified as stars.}. It has the following analytical representation

\begin{equation}
\label{eq:KroupaBD}
\xi_{BD}(m)=\frac{k}{3}\cdot\left(\frac{m}{0.07}\right)^{-0.3\pm0.4}, \hspace{3.5cm} 0.01M_{\odot} < m \leq 0.15 M_{\odot},
\end{equation}
\begin{align}
\label{eq:KroupaStars}
\xi(m)=& k\cdot \left(\frac{m}{0.07}\right)^{-1.3\pm0.3}, \hspace{3.5cm} 0.07M_{\odot} < m \leq 0.5 M_{\odot},\nonumber\\
\xi(m)=& k\cdot \left(\frac{0.5}{0.07}\right)^{-1.3\pm0.3}\cdot  \left(\frac{m}{0.5}\right)^{-2.3\pm0.36}, \ \ \ \ 0.5 M_{\odot} < m \leq 150 M_{\odot},
\end{align}
where $k$ is a constant.
 
 \citet{Thies2007} explored the parametric space of the  IMF for the BD population. In specific, they performed a $\chi^2$ fit of the maximum BD mass ($m_{max,BD}$), the slope of the BD IMF ($\alpha_{BD}$ in Eq. \ref{eq:KroupaBD}), and the population ratio ($\mathcal{R}_{pop}$), which is the ratio of individual BD to individual stars. They performed this fit on literature data from the Trapezium, Taurus, IC 348 and the Pleiades nearby clusters. Figure \ref{fig:IMFThies2007}, which reproduces their Figure 7, shows their results in these four clusters. In their work, the stellar IMF is the canonical one (Eq. \ref{eq:KroupaStars}). In the case of the Pleiades and IC 348, the slope of the BD IMF remained the canonical value $\alpha_{BD}=0.3$ due to the sparse data, and the Pleiades  incomplete data particularly. The resulting system IMF is also shown as the solid curve. It results from the addition of the two IMFs (BD and stars) but without an overlap or discontinuity between them \cite[see][for details]{Thies2007}.
 
 \begin{figure}[htbp]
\begin{center}
\includegraphics[width=0.8\textwidth]{background/Figures/F7_Thies2007.pdf}
\caption{Observed mass distribution (histograms and dots) of the Trapezium, Taurus, IC348 and the Pleiades nearby clusters (from top to bottom). Also shown the canonical stellar (long dashed line), fitted BD (dotted), and system (solid curve) IMFs of \citet{Thies2007} . Reproduced from Figure 7 of \citet{Thies2007}, \textit{\usebibentry{Thies2007}{title}}, \usebibentry{Thies2007}{journal}, \usebibentry{Thies2007}{volume}}
\label{fig:IMFThies2007}
\end{center}
\end{figure}
 
As shown in this Section, the IMF has been the subject of several studies which have transformed our knowledge about it. This knowledge evolved from the simple and efficient model of \citet{Salpeter1955} in the high mass range, to the more complicated models of \citet{Thies2007, 2013pss5.book..115K} for the BD and low-mass stars. Although these models currently explain many of the observed phenomena, our knowledge of the physical processes that lead to the observed IMF is still incomplete. In an attempt to address the latter, the numerical simulations of the star formation process have done great progress. The next section is a tiny review of what these simulations have achieved in the past years.


\section{Numerical simulations of the early stages of star formation}
\label{sect:numerical_simulations}

In the first decade of this century, numerical simulations of star forming regions have proved to be of paramount importance in the decoding the very early stages of the star formation process \cite[e.g][]{2003MNRAS.339..577B,2005A&A...435..611J,2009MNRAS.392..590B,2009MNRAS.392.1363B,2009MNRAS.397..232B}. For example, \citet{2003MNRAS.339..577B} using smooth particle hydrodynamics were able to simulate the collapse and fragmentation of a large-scale ($50 M_{\odot}$ within 0.375 pc radius) turbulent molecular cloud to from a stellar cluster. During the very first 0.1 Myr, which was the time covered by their simulation, they were able to simultaneously form discs and binary stars. The cloud formed roughly equal numbers of stars and BD (23 and 27, respectively) resulting in a mass distribution with a flat slope in the range $0.01-0.5 M_{\odot}$; see Fig. \ref{fig:IMFBate2003}. \citet{Offner2014} provides a review of the stellar initial mass distribution (function), and of the physical effects included in numerical simulations (radiative feedback, competitive accretion, dynamical interactions, magnetic fields) particularly. 

In recent years, the works of \citet{2015ApJ...815...27K} and \citet{2015MNRAS.452..566B}, using the cold collapse paradigm (neglecting magnetic fields, radiative transfer and feedback), were able to probe that the main source driving the star formation process is gravity. Their simulations were typically run until 0.85 Myr in a box of 3 pc of side, and with masses in the few thousands of $M_{\odot}$. In particular, the mass distribution obtained by \citet{2015ApJ...815...27K} reproduces reasonably well the high mass range of the current models of the initial mass distribution, the IMF of \citet{Chabrier2005} particularly (see Fig. \ref{fig:IMFKuznetsova}). However, compared to this IMF, they produce too few low-mass stars and brown dwarfs.

% Remaining issues

Numerical simulations have proven to be of great use in the understanding of the star formation process, in the very early phases ( $\leq$1 Myr) of its evolution particularly.  Despite the fact that many of these simulations are in agreement with the observed mass distributions, currently they: i) do not incorporate all astrophysical effects, ii) resolve close binaries and multiple systems, and iii) produce enough stellar objects to improve the statistics \citep{Offner2014}. Furthermore, these simulations require input from the observations (e.g. density, turbulence and magnetic fields in the molecular cloud). More importantly, they require feedback from the observations in order to fine tune the parameters of the physical processes. Real and simulated data must be compared in order to improve the latter. Thus, to constrain the current theories of the star formation process, precise and detailed studies of the early phases of cluster formations are still required. This is the objective of the DANCe project, from which the present work is part of.


% Why are these issues important?
\begin{figure}[ht!]
\begin{center}
\includegraphics[width=0.8\textwidth]{background/Figures/F10_Bate2003.pdf}
\caption{Mass distribution resulting from the numerical simulation of \citet{2003MNRAS.339..577B}. The lines show the mass distributions of \citet{Salpeter1955}, \citet{1979ApJS...41..513M} and \citet{2001MNRAS.322..231K}. Reproduced from Figure 10 of \citet{2003MNRAS.339..577B}, \textit{\usebibentry{2003MNRAS.339..577B}{title}}, \usebibentry{2003MNRAS.339..577B}{journal}, \usebibentry{2003MNRAS.339..577B}{volume}}
\label{fig:IMFBate2003}
\end{center}
\end{figure}

\begin{figure}[ht!]
\begin{center}
\includegraphics[width=0.8\textwidth]{background/Figures/F11_Kuznetsova2015.pdf}
\caption{Mass distribution resulting from the numerical simulation of \citet{2015ApJ...815...27K}. The High (solid line) and Low (dashed line) resolution simulations reach $0.05\, M_{\odot}$(solid vertical line) and $0.15\, M_{\odot}$ (dashed vertical line), respectively. Also shown the normalised IMF of \citet{Chabrier2005} (red dashed line). Reproduced from Figure 11 of \citet{2015ApJ...815...27K}, \textit{\usebibentry{2015ApJ...815...27K}{title}}, \usebibentry{2015ApJ...815...27K}{journal}, \usebibentry{2015ApJ...815...27K}{volume}}
\label{fig:IMFKuznetsova}
\end{center}
\end{figure}

\section{The DANCe project}
\label{sect:DANCeproject}

The DANCe (Dynamical Analysis of Nearby Clust\emph{e}rs) project is an international and multidisciplinary collaboration of researchers under the leadership of Hervé Bouy. The main objective of the this project is \textsl{unravelling the origin of the mass function} of stellar Nearby Young Clusters (NYC). This objective demands a thorough compilation of resolved star forming populations over the entire mass spectrum and in diverse star forming environments. In specific, it requires complete catalogues of the star cluster members, and only those members. Disentangling star cluster members from the field population is a titanic task that the astrophysical community is on the verge of complete thanks to the future \emph{Gaia} \citep{GAIA} data. However, even with Gaia, the development and success of star formation theories will be pending a comprehensive identification of the sub-stellar populations and of the dynamical content of embedded cluster cores. \emph{Gaia} will leave aside the most massive BD and planets, reaching only down to 0.03 $M_{\odot}$ on the best cases \citep{Sarro2013}. Furthermore, \emph{Gaia} reaches the 20 mag limit in the $G$ band \cite[350 nm to 1000 nm][]{2010A&A...523A..48J} what makes it almost blind to the dust and gas in which the majority of the star forming regions are still embedded. 

The DANCe photometric and astrometric measurements will complement \emph{Gaia}'s exquisite astrometric precision in a selected list of star forming regions; see Tables \ref{tab:maintargets} and \ref{tab:secondarytargets} for the main and secondary targets of the DANCe project, respectively. A pilot study of the DANCe data properties has already been conducted on the Pleiades cluster. For more details on it I refer the reader to Section \ref{sect:DR2} and to \citet{Bouy2013}.

\begin{table*}
\caption{Main targets}      
\label{tab:maintargets}    
\centering                     
\begin{tabular}{l c c c c c c}  
\hline
\hline
Name & R.A.  & Dec  & Distance & Age & $\mu_{\alpha} cos(\delta)$ & $\mu_{\delta}$ \\
  & (J2000) & (J2000) & (pc) & (Myr) & (mas/yr) & (mas/yr) \\
\hline
Pleiades                &  56.750     &  +24.117   	 &   120               &    120           &  19                                       &  -44                          \\
 IC2391             &  130.133    &  -53.033   	 &   175               &    50            &  -25                                      &  23                           \\
 IC2602             &  160.742    &  -64.400   	 &   160               &    50            &  -22                                      &  10                           \\
 $\gamma^{2}$Vel    &  122.383    &  -47.336   	 &   350               &    5$\sim$10     &  -6                                       &  10                                                 \\
 NGC 2547    &  122.6071     &  -49.1675   	 &   450               &    20$\sim$35     &  -6                                       &  10                           \\
 IC 348   &  56.142   &  +32.163 	 &   $\sim$300  &  2   &  7    &  -9   \\
 NGC 1333   &  52.258   &  +31.348 	 &   $\sim$300   &  1   &  ?    &  ?   \\
 CrA   &  285.462    &  -36.982  	 &   130    &  1   &  ?    &  ?   \\
 $\eta$-Cha   &  130.525     &  -79.027  	 &   97    &  5$\sim$9   &  -30    &  28   \\
 $\epsilon$-Cha   &  179.90657  &  -78.22184  	 &   111    &  5$\sim$9   &  -38    &  -1   \\
 Upper Scorpius   &  243.0     &  -23.4   	 &   120    &  5$\sim$10   &  -9    &  -25   \\
 Lupus I   &  235.7587      &  -34.1517  	 &   140    &  1$\sim$3   &  $\sim$-15    &  $\sim$-25   \\
 Lupus II   &  239.283       &  -37.785  	 &   140    &  1$\sim$3   &  $\sim$-15    &  $\sim$-25   \\
 Lupus III   &  242.40       &  -39.05   	 &   140    &  1$\sim$3   &  $\sim$-15    &  $\sim$-25   \\
 Serpens   &  277.487    &  +01.240 	 &   $\sim$400    &  1$\sim$3   &  ?    &  ?   \\
 Ophiuchus   &  247.025   &  -24.542 	 &   125    &  1$\sim$3   &  -6    &  -25   \\
 Taurus   &  70.25    &  +25.87 	 &   140    &  1$\sim$3   &  8    &  -21   \\
 Blanco 1   &  1.029    &  -29.833  	 &   250    &  $\sim$100   &  21    &  3   \\
 Orion   &  83.8221     &  -05.3911  	 &   450    &  1$\sim$3   &  1.7    &  -0.3   \\
 NGC2451A   &  116.35       &  -37.97 	 &   197    &  50$\sim$80   &  -22    &  14   \\
 NGC2451B   &  116.35       &  -37.97 	 &   360    &  50$\sim$80   &  -10    &  14   \\
 CygOB2   &  308.30      &  +41.32  	 &   1450    &  1$\sim$7   &  -1.6    &  -4.7   \\
 IC4665   &  266.575     &  +05.717  	 &   350    &  $\sim$35   &  -0.5    &  -7   \\
 Collinder 359   &  270.275      &  +02.900  	 &   450    &  $\sim$100   &  -0.4    &  -8.   \\
 $\lambda$-Ori   &  83.775    &  +09.933  	 &   450    &  3$\sim$5   &  0.8    &  -3.   \\
 $\sigma$-Ori   &  84.675     &  -02.600  	 &   450    &  3$\sim$5   &  1.7    &  0.5   \\
 Cha I   &  166.700       &  -77.300  	 &   160$\sim$180    &  1$\sim$3   &  -21    &  1   \\
 Cha II   &  193.4      &  -77.2   	 &   160$\sim$180    &  1$\sim$3   &  -24    &  -8   \\
 Cha III   &  189.4142     &  -80.2533   	 &   160$\sim$180    &  1$\sim$3   &  ?    &  ?   \\
 NGC2516  &    119.51667  &  -60.75333	 &  410   &  135   &  -4   &  11    \\
\hline
\end{tabular}
\end{table*}


\begin{table*}
\caption{Secondary targets}      
\label{tab:secondarytargets}    
\centering                     
\begin{tabular}{l c c c c c c}  
\hline
\hline
Name & R.A.  & Dec  & Distance & Age & $\mu_{\alpha} cos(\delta)$ & $\mu_{\delta}$ \\
  & (J2000) & (J2000) & (pc) & (Myr) & (mas/yr) & (mas/yr) \\
\hline
NGC 752            &  29.421     &  +37.785   	 &   460               &    1000          &  8                                        &  -12                          \\
 NGC 6774            &  289.175     &  -16.283   	 &   $\sim$300               &    3000          &  -1                                        &  -29                          \\
$\alpha$-Per   &  51.079     &  +49.862  	 &   171$\sim$200    &  50$\sim$70   &  22    &  -25   \\
 Praesepe   &  130.100    &  +19.667  	 &   600$\sim$700    &  1$\sim$3   &  ?    &  ?   \\
 NGC 2264   &  100.242   &  +09.895  	 &   920    &  3$\sim$5   &  -0.6    &  4.   \\
M35   &  92.225     &  +24.333   	 &   850    &  $\sim$130   &  2.7    &  -3.5   \\
 M67   &  132.825      &  +11.800   	 &   860    &  3200$\sim$5000   &  -6.5    &  -4.5   \\
 NGC 6791   &   290.221    &  +37.772   	 &   4100    &  $\sim$4300   &  4.0    &  0.6   \\
 Coma Berenices   & 185.6262   & +25.8450	 &   96    &  $\sim$400   &  -12.4   &  -9.4   \\
 M39   &  322.950   &  +48.433 	 &  325   &  250   &  -8   & -20      \\
 Trumpler 10   &  131.975   &  -42.450 	 &  425   &  180   &  -12   &  7   \\
 Collinder 285   &  220.22   &  +69.57 	 &  25   &  200   &  ?   &  ?   \\
 M41   &   101.504   &  -20.757 	 &  240   &  700   &  ?   &  ?   \\
 Stock2   &   33.679   &  +59.485 	 &  300   &  100   &  15   &  15    \\
 $\mu-$Oph group   &   264.46129  &  -08.1188 	 &  120   &  -12   &  -21    \\
 Upgren 1   &   188.75417  &  36.36667 	 &  110   &  ?   &  -98   &  -49    \\
 \hline
\end{tabular}
\end{table*}


 

Although accurate, precise and complete data sets of NYC are the foundation of the DANCe project. The unravelling of the mass function requires also an statistical methodology able to deal not just with the properties of the data sets and the particularities of each NYC. It must be able to disentangle the few hundreds cluster members from within the millions of sources of the field population. Working with individual stars one by one is no longer possible. Under the current paradigm of artificial intelligence, the intelligent systems offer a viable solution to the classical needle in a haystack problem. The DANCe project has already successfully tested its first intelligent classifier. Its methodology and astrophysical results on the Pleiades clusters are detailed in the works of \citet{Sarro2014} and \citet{Bouy2015}, respectively. 

With both the precise and accurate data and methodology, the DANCe project will be able to not just complement \emph{Gaia} in the low-mass regime, but also deliver a complete inventory of the early phases of cluster formation in different environments. It will unravel the roles of the environment and initial condition in the final mass distribution, which is the ultimate goal of the DANCe project.

The present work is the second generation of the DANCe intelligent systems (ISs). This new IS minimises the biases of the previous one and those present in most of the current classifiers in the literature. It steps ahead of the classification paradigm, the current one in the literature, and focuses on a new one: \textit{learning the kinematic and photometric distributions of the cluster populations}, both single and binary stars. The classification task is now a by-product of this new IS. In the following section, I describe the properties of the previous DANCe IS classifier \cite[the one from][]{Sarro2014}, together with the current ones from the literature.    

\section{Methodologies for the classification of cluster members}
\label{sect:current_methodologies}

As mentioned above, the vast majority of the past and current methodologies for the analysis of stellar clusters work under the classification paradigm. Given the objects observables and the current knowledge of the cluster, they assign cluster membership probabilities. These are then used to classify the objects into the field or cluster populations. The classification depends on a probability threshold, which beside imposing a trade-off between completeness and contamination in the resulting sample, is not always established in an objective way.  

In his pioneer work, which was also conducted in the Pleiades cluster, \citet{Trumpler1921} selected candidate members using three criteria: 
\begin{enumerate}
\item projected distance to the cluster centre,
\item similarity of the proper motions to those of the cluster, and
\item the degree of accordance with the relation between colour and brightness of the cluster. 
\end{enumerate}

During almost one century since the publication of Trumpler's work, his three criteria remain valid and are still widely used.
 
\citet{Vasilevskis1958} and \citet{Sanders1971} put a solid probabilistic background to Trumpler's second criterium. \citet{Vasilevskis1958} expressed (in their Eq. 4) the probability of an object to belong to the cluster as 

\begin{equation}
p_c = \frac{N_c \cdot \Phi_c(\lambda,\beta)}{N_c \cdot \Phi_c(\lambda,\beta) + N_f \cdot \Phi_f(\lambda,\beta)}, \nonumber
\end{equation}
with $N_c$ and $N_f$ the number of cluster and field stars, respectively. $\Phi_c$ and $\Phi_f$ represent the bivariate normal distributions of cluster and field, respectively, and $\lambda,\beta$ are the proper motions in galactic coordinates. The selection of galactic coordinates and the symmetry of the proper motion uncertainties allowed them to assume that the cluster bivariate normal distribution, $\Phi_c$,  had circular symmetry (i.e. a diagonal covariance matrix). They fitted the means of these bivariate normal distribution by eye using histograms for each coordinate. \citet{Sanders1971} used the same procedure of  \citet{Vasilevskis1958}, but fitted the parameters using the maximum-likelihood principle. This methodology remained almost intact during the following decades. 

Later, \citet{1979PVatO...1....1D} included the spatial information (Trumpler's first criterium) into the membership probability determination. Although his work is not available on-line, his formulation is given in the work of \citet{1995AJ....109..672K}. In the latter, the cluster membership probability of an object, given its positions and proper motions, is 

\begin{equation}
P_{\mu,\alpha,\delta}= \frac{N_c\cdot S_c(\alpha,\delta)\cdot \Phi_c(\mu_{\alpha},\mu_{\delta})}{N_c\cdot S_c(\alpha,\delta)\cdot \Phi_c(\mu_{\alpha},\mu_{\delta}) +N_f\cdot S_f(\alpha,\delta)\cdot \Phi_f(\mu_{\alpha},\mu_{\delta})}, \nonumber
\end{equation}
with $\alpha$ and $\delta$ the stellar positions in equatorial coordinates, and $S_c$ and $S_f$ the cluster and field spatial (positional) distributions. \citet{1995AJ....109..672K} assumed exponentially decaying and uniform distributions for the cluster and field spatial distributions, respectively. To model the proper motion distributions they also used bivariate normal distributions.

In the first decade of this century, authors started to incorporate Trumpler's third criterium in the selection of members. For example \citet{2003A&A...400..891M, 2004A&A...416..125D,2007A&A...470..585B,Lodieu2012} used hard cuts in the colour, or in the colour magnitude diagrams (CMDs) to discriminate candidates. \citet{2003A&A...400..891M} used conservative cuts based on the theoretical isochrone models of \citet{1998A&A...337..403B}, while \citet{2004A&A...416..125D,2007A&A...470..585B,Lodieu2012} used arbitrary cuts in the CMDs diagrams. However, despite the conservative or objective these cuts may be, as pointed out by \citet{Sarro2014} they have at least two drawbacks. They render membership probabilities i) which are inconsistent across magnitude bins, and ii) only for those sources with measurements in the CMDs used for selection.
  
The previous works incorporated most of the observables (positions, proper motions and photometry) available for the vast majority of the stars. Other useful observables for the discrimination of cluster members are the parallax, the radial velocities and other indicators of youth \cite[e.g. lithium abundance, photospheric activity, rotation, see for example][]{2016A&A...596A.113B}.  Although these observables can be of great use, they are only available for limited samples of objects. Furthermore, they tend to be biased towards the brighter objects. For these reasons, the members selection has been done using the astrometric and photometric observables: positions, proper motions and photometric bands.

In recent years, the astrophysical community started to develop automated methodologies for the cluster members selection. This developments has been motivated by two main reasons. First, computing power has become widely available thus allowing the implementation of more complicated and computing demanding techniques. Second, the arrival of automated surveys providing incredible amounts of data for millions of objects have rendered obsolete the previous object-by-object oriented techniques. In the following, I will review the methodologies of \citet{Malo2013,Gagne2014,Riedel2017} for stellar associations, and the ones of \citet{KroneMartins2014,Sarro2014,Sampedro2016} for star clusters. Although this work focuses only on the analysis of star clusters, the methodologies applied to associations are related and some times also applied to them. Thus, I review them as well.

\citet{Malo2013} develop a code for the Bayesian Analysis of Nearby AssociatioNs (BANYAN). It establish membership probabilities to seven nearby young moving groups ($\beta$ Pictoris, Tucana-Horologium, AB Doradus, Columba, Carina, TW Hydrae, and Argus) using a naive (no correlations included) Bayesian classifier. Their classification uses kinematic and photometric models of the seven moving groups and the field. These models are constructed with the previously known \emph{bona fide} members and field objects. The photometric model in the low-mass range is extended beyond the known members using evolutionary models. One of the great advantages of this work is its ability to simultaneously obtain membership probabilities for the seven nearby young moving groups. It works on a six dimension observable space: $I_c$ and $J$ bands, proper motions and stellar positions. The recovery (True positive) and false alarm (False positive) rates of this classifier are 90\% and 10\%, respectively; the latter varying according to the moving group.

\citet{Gagne2014} improve \citet{Malo2013} methodology by developing BANYAN-II. It is tuned to  identify low-mass stars and brown dwarfs by the use of redder photometric colours from \emph{2MASS} and \emph{WISE}. The spatial and kinematic distributions are modelled with no alignment to the galactic coordinates as is done by BANYAN. They also included the observational uncertainties and the ability to deal with missing values, at least in parallax and radial velocities. When the parallaxes and radial velocities are not present, the estimated contamination rate ranges between 20\% and 80\% for a wide range (0.2-0.8) of probability cuts (see their Figure 5).

\citet{Riedel2017} developed a kinematic membership analysis code for LocAting Constituent mEmbers In Nearby Groups (LACeWING). Using only the spatial and kinematic information of stars, LACeWING, computes membership probabilities to 13 nearby young moving groups (NYMGs). LACeWING computes what they call the goodness-of-fit value, which is the distance between each object observables and the value of the NYMG divided by the quadratic sum of the uncertainties. Then it draws synthetic stars, compute their goodness-of-fit values, and bin them. The membership probability of a real object correspond to the the fraction of synthetic stars, in the same goodness-of-fit bin, that were generated as members of the moving group. These authors do not provide estimates of contamination and recovery rates on synthetic samples. Instead, they directly give the number of recovered and false positives returned by LACeWING in each NYMG. As mentioned by these authors, the recovery rates are similar to those of BANYAN and BANYAN-II. 


\citet{KroneMartins2014} created the Unsupervised Photometric Membership Assignment in Stellar cluster code (UPMASK). It establishes cluster membership probabilities in a frequentist approach using positions and photometry. It is an unsupervised and data driven iterative algorithm. Their methodology relies on clustering algorithms and the principal component analysis (PCA). It establishes membership probabilities by randomly generating synthetic stars based on the individual uncertainties. The membership of an object is the fraction of randomly generated stars in which the object was classified as a cluster member. Although untested, the authors mention that their methodology is able to incorporate proper motions, deal with missing data and with different uncertainty distributions. Measured on synthetic samples (with typical sizes ranging from thousands to tens of thousand objects), these authors find that UPMASK delivers true positive (recovery) rates grater than 80\% for a probability classification threshold of p=0.9 (they do not provide any reason for the use of this particular value). One of the great advantages of  UPMASK is its ability to deal with superimposed clusters. 

\citet{Sarro2014} created a Bayesian classifier to infer posterior membership probabilities based only on proper motions and photometry. Their cluster model is data driven and includes some parametric correlations (it is not a naive classifier like BANYAN or BANYAN-II). They construct the proper motions and photometric models of the field and cluster with clustering algorithms. However, the photometric model of the cluster is constructed using the principal curve analysis. Their treatment of uncertainties and missing values is consistent across observed features. They infer the best parameter values of both cluster and field models using a Maximum-Likelihood-Estimator (MLE) algorithm, but based only on completely observed objects (i.e. without missing entries). Afterwards, they assign membership probabilities to objects with missing entries. They found an optimal probability classification threshold at $p=0.85$ using synthetic data sets of 2 million sources. At this threshold they measure a contamination rate of 4.5\% and a true positive rate of 92.9\%.

\citet{Sampedro2016} created a Bayesian algorithm to estimate membership probabilities to open clusters. They compute the euclidean distance, in a multidimensional normalised space, of each object to the over-density of the cluster. Then, they assign membership probabilities according to this distance assuming univariate normal distribution for both cluster and field. Their methodology assumes that the cluster members are more densely concentrated than the field population. Since they use an euclidean metric to compute distances in a normalised space (normalised by the individual uncertainties), their methodology is highly affected by the heteroscedasticity of the presence of missing values in the observables. In an euclidean metric, objects with missing values have smaller, or at most equal, distances than those of objects with complete observations. They use a 0.5 probability classification threshold based on the Bayes minimum error rate, however, they do not provide error rates. On synthetic data sets of 500 objects they measure completeness and misclassification as a function of  the similarity between the cluster and field univariate normal distributions. They do this varying i) the fraction of cluster members in the total data set size (from 20\% to 80\%), and ii) the number of variables (from among positions and proper motions). Their misclassification is in the range of 25\% to 5\% for two to four variables, while their completeness is better than 90\%. 

The previous methodologies perform well on the classification task, and successfully led to the identification of many new high probability members of nearby clusters and associations. However, key aspects still need to be tackled. These are 
\begin{itemize}
\item Uncertainty of membership probabilities. These, as any other measurement, are uncertain. The previous authors report just the first moment of the object membership probability distributions.
\item Treatment of objects with missing entries in their observables. In astronomy, measurements are strongly affected by the brightness of the source. The physical limits imposed by the dynamical range of detectors lead to non-uniform distributions of objects with missing entries. At the bright end, the excess of photons saturates the detector pixels and, most of the time, renders the measurements useless \cite[however, see][ for examples of high-precision astrometry and photometry on saturated images]{2003hstc.conf..346M,2013AJ....146..106O}.  At the faint end, sources with brightness below the limit of sensitivity are not detected. Furthermore, in a multi-wavelength data set the intrinsic stellar colours add another level of correlation between the sensitivity limits of different bands. The previous effects create a non-uniform distribution of objects with missing values. In addition, missing values can also appear due to other non-random effects (e.g. proximity to a bright source), which will further affect any analysis that discards them. If the distribution of objects with missing entries will be uniform, then results based only on the completely observed (non-missing value) objects would be unbiased. However, the less uniform the distribution of missing-value objects is, the most biased the results are. See Chap. 8 of \citet{Gelman2013} for a discussion on missing at random and ignorability of the missing pattern. Objects with missing values have an important effect in the resulting samples of members. The cluster and field model are conditioned on the objects used to construct them. Since objects with missing values are not included in these models, the resulting samples of members may be biased.
\item Heteroscedastic uncertainties. Each object in the cosmos is unique. Thus assuming that they are alike is useful in certain situations. In general, homosedasticity is assumed for practical reasons. However, it may lead to biases. For example, it is known that the performance of PCA \cite[used by][]{KroneMartins2014}, and consequently the Principal Curve Analysis \cite[used by][]{Sarro2014}, are strongly affected by heteroscedastic uncertainties \citep{Hong2016}. 
\item Sampling bias. Using a probability classification threshold to select candidate members results in a trade-off between contamination and completeness. To keep contamination under reasonable levels samples are never complete. Therefore, any further statistical analysis resulting from them suffers from a sampling bias.
\item Propagation of uncertainties. Uncertainties in the observables must be propagated until the very end of subsequents astrophysical analyses. Currently, once a list of candidate members have been obtained, luminosity functions and then mass functions are derived on the basis of poisson uncertainties. At best, the uncertainties include the photometric ones, but completely discard proper motion uncertainties and those of the model that generated the list of candidates.
\end{itemize}

The methodology presented in this work attempts to address the previous problems.

\section{A new intelligent system: the Bayesian Hierarchical Model}
\label{sect:newIS}

As mentioned before, this new IS steps ahead of the classification paradigm and focuses on a new one: \textit{learning the kinematic and photometric distributions of the cluster populations}, both single and binary stars. This new IS addresses the aforementioned issues in the following way.

\begin{itemize}
\item Uncertainty of membership probabilities. Together with the kinematic and photometric distributions of the cluster population, it renders the cluster and equal-mass binaries (full) membership probability distributions, not just point estimates of them. Thanks to its Bayesian framework, it computes the probability distribution of the class, cluster vs field and single vs binary, given the object datum. 

\item Treatment of objects with missing entries in their observables. Thanks to its Bayesian framework, the new IS is able to include in its data set objects with missing entries in its observables. The missing values are treated as parameters which are marginalised with the aid of a prior.

\item Heteroscedastic uncertainties. In the new IS, it is assumed that objects uncertainties share the same family distribution, the multivariate normal. However, the values of the uncertainties are not assumed to be the same. The new IS incorporates the intrinsic heteroscedasticity of the data. It assumes that the observed data results from the addition of a noise process to the \emph{true}\footnote{The true data is that which would be observed under negligible uncertainties.} data. The true intrinsic underlying relation is accessible after deconvolution of the observed uncertainties \cite[see][for another example of deconvolution]{2009ApJ...700.1794B}.

\item Sampling bias. The new IS avoids the bias associated to the use of only the high membership probability objects in the following way. It derives the kinematic and photometric distributions of the cluster populations taking into account each data set object proportionally to its cluster likelihood. Thus, there is no need of a probability classification threshold to obtain the clusters distributions.

\item Propagation of uncertainties. Thanks to its Bayesian framework and the use of heteroscedastic uncertainties, the new IS is able to propagate the observational uncertainties directly into the posterior distribution of the parameters of the cluster model. Then, the distributions of the parameters can be propagated into the luminosity and mass distributions. The resulting mass distribution incorporates not just the observational uncertainties of all the objects, but also the uncertainties associated with the cluster model. 
\end{itemize}

The core of this new IS is the Bayesian framework. The latter demands the use of prior distributions for the value of its parameters. To avoid as much as possible the subjectivity of choosing them, the new IS uses the Bayesian Hierarchical Model (BHM) methodology together with weakly informative priors  \cite[see]{Gelman2006,Gelman2008,Huang2013,Chung2015}. The works of \citet{Jefferys2007,Shkedy2007,Hogg2010,Sale2012, Feeney2013} are some examples of the use of BHMs in astrophysics. In the BHMs, the parameters of the prior distributions are given by other distributions in a hierarchical fashion. These parameters are also fitted from the data. Thus, BHM provide the most objective approach to the settling of prior distributions \citep{Gelman2006}. However, it comes at a price: BHMs are computationally more expensive because they require far more parameters than standard approaches. 

In such a way, the new IS learns the posterior distributions of the parameters in the BHM given the kinematic and photometric data. Due to the high dimension of the parametric space (the BHM has 85 parameters), the learning process demands a fast and reliable technique to obtain the posterior distributions of the parameters. Furthermore, the typical data set size for a cluster and the surrounding field is in the order of millions of sources. Thus the likelihood of the data must loop over all these millions of sources. The high dimensionality of the parametric space and the large data set size demand a technique that, on top of being fast and reliable, will also be compatible with parallel computing. Thus reducing to a minimum the computing time. The Markov Chain Monte Carlo techniques offer the demanded requirements. In specific, the \emph{emcee} algorithm \citep{Foreman2013} was chosen due to  its excellent performance properties and its ability to work in a parallel computing scheme.  

The BHM presented in this work has been benchmarked on the Pleiades cluster, with the DANCe Data Release 2 \cite[DR2,]{Bouy2015} particularly. This reason roots in the following points. First, the Pleiades cluster together with the Orion Nebula Cluster are probably the most studied clusters in the history of astronomy. Thus, the current knowledge of Pleiades makes it ideal to benchmark any new tool or theory. Second, currently the DANCe project has processed the data of only three clusters: M35, CygOB2 and the Pleiades. CygOB2 and M35 are relatively poorly understood compared to the Pleiades, and are far from the sun (see Tables \ref{tab:maintargets} and \ref{tab:secondarytargets}), which give them small proper motions values. Since the Pleiades is a well understood cluster, with well characterised lists of members \cite[e.g][]{Stauffer2007, Lodieu2012, Sarro2014, Bouy2015} and precise and accurate measurements, it is the perfect case study for the development, testing and validating of the new IS described here.

Once the BHM is constructed, it was applied on synthetic data sets to analyse its performance as a classifier. The results of this analysis show that at an optimal probability threshold of $p=0.84$ the BHM has a True Positive Rate (recovery rate) of $90.0\pm0.05$\% and a contamination rate of $4.3\pm0.2$\%. The area under the receiving-operating characteristic curve (AUC) is 0.99 which indicates that it as an excellent classifier. The synthetic analysis reveals that the expected value of the contamination in the kinematic and photometric distributions is $5.8\pm 0.2$\%.
  
The application of the BHM to the Pleiades DANCe DR2 data set delivered the kinematic an photometric distributions of the cluster members. As a by-product, it also delivered cluster and EMB membership probabilities for all objects in the data set. From this analysis, the following results are remarkable.

\begin{itemize}
\item The recovered membership probabilities show an outstanding agreement (99\%) with the previous results of \citet{Bouy2015}. Also, the BHM recovers 91\% of \citet{Rebull2016} candidate members. This figure is slightly better than the expected $90.0\pm0.05$\% true positive rate at the probability classification threshold of $p=0.84$.
\item There are 205 new candidate members, which represent 10\% of the cluster population. This 10\% is a proxy of the knowledge discovery obtained by the new IS.
\item The derived luminosity distributions in the infrared bands ($Y,J,H, \rm{and} K_s$) and in the completeness interval are in good agreement with the previous ones of \citet{Bouy2015}.
\item The derived system mass distribution \cite[using an age of 120 Myr and the BT-Settl isochrone model of][]{Allard2012} shows a general agreement with the IMFs of \citet{Chabrier2005} and \citet{Thies2007}, in the intermediate mass range specifically. However, in the low-mass and BD domains, the IMFs predict to many low-mass stars and BD. 
\end{itemize}

By minimising bias, propagating uncertainties and doing a comprehensive modelling of the data particularities (missing values and heteroscedastic uncertainties) the BHM, developed, tested and validated in the present work, delivers the kinematic and photometric distributions of the cluster single and equal-mass binary populations. This new intelligent system represents the state of the art in the statistical analysis of Nearby Young Clusters.

The rest of this work is structured as follows. In Chapter \ref{chap:pleiades}, I do a compilation of the current knowledge on the Pleiades cluster. In particular, that concerning its distance, spatial, velocity, luminosity and mass distributions. Also I give details of the Pleiades DANCe DR2 data set. The incredibly detailed knowledge of the Pleiades cluster and the high precision and carefully designed observations of the DANCe project are the main reasons for the benchmarking the BHM on this cluster. Later, in Chapter \ref{chap:BHM}, I give a brief introduction to probability theory and also present the details of the BHM methodology and of the techniques used to sample the statistical distributions of the cluster populations. Later, in Chapter \ref{chap:Results}, I give details of the analysis performed on synthetic data, of the comparison of the recovered candidate members with those from the literature, of the kinematic and photometric distributions, and of the present day system mass distribution.  Finally, in Chapter \ref{chap:conclusions}, I present my conclusions and my point of view concerning the work that must be done in the near future to continue improving our knowledge of the star formation process.



