%!TEX root = ../thesis.tex
\chapter{Introduction}
\label{chap:introduction}

%\nocite{*} % remove this line to add only cited references

The majority of the stars form in clusters. \citet{2000AJ....120.3139C} reports that $50-70\%$ of young ($\leq10Myr$), and $25-70\%$ of the relatively old ($\sim100Myr$), stellar populations were formed in clusters. \citet{2003AJ....126.1916P} and \citet{2003ARA&A..41...57L} find that 80\% and 90\%, respectively, of the stars form in clusters with more than 100 members. Furthermore, as indicated by  \citet{2003AJ....126.1916P}, these large clusters represent  22\% of the star forming regions. With only 8\% of the stars forming in regions with five to 30 members. However, from these large ($\geq 100$ members) clusters, only less than 7\% will survive as bounded clusters when reaching an age of $120 Myr$ \citep{2003ARA&A..41...57L}. The remaining of the star forming regions will become unbounded and their stars will freely populate the galaxy. Thus, to understand the general rules that govern how the majority of stars form, as well as the properties of the stars that populate our galaxy, it is crucial to fully decode the formation and evolution of stellar clusters. 

    In the first decade of this century, numerical simulations of star forming regions have proved to be of paramount importance in the decoding the very early stages of the star formation process \citep{2003MNRAS.339..577B; 2004ApJ...605..800L;2005A&A...435..611J,2009MNRAS.392..590B,2009MNRAS.392.1363B,2009MNRAS.397..232B}. For example, \citet{2003MNRAS.339..577B} using smooth particle hydrodynamics simulations were able to reproduce the collapse and fragmentation of a large turbulent molecular cloud ($50 M_{\odot}$ within $0.375pc$ radius), which formed about 50 stars.  

Spectacular advances have been made in recent years in the field of stellar cluster formation. New statistical theories of star and planet formation have replaced the standard Shu (1987) model of isolated star formation and offer a much more dynamic view of prompt multiple fragmentation driven by MHD supersonic turbulence in molecular clouds (see Fig. 1, and e.g. Bate et al. 2003; Li et al. 2004; Jappsen et al. 2005). Smooth particle hydrodynamics (SPH) calculations are now able to simulate the collapse and fragmentation of large turbulent molecular clouds forming several hundreds of stars (e.g. Bate 2009).
However, the outcome of these simulations depend crucially on the physical parameters of the parent molecular cloud (density, turbulence), and if radiative feedback (Bate 2009) and magnetic field (Hennebelle \& Teyssier 2008) are taken into account. In particular, important open issues remain:

Description of:

- Motivation to investigate nearby young open clusters.

- The importance of the IMF

- Current state of knowledge in dynamical simulations.

- The problematic of constraints in dynamical theories.

It must be clear which is the objective

Description of the Nearby open clusters and their properties.

\section{The DANCe project}

- List of open clusters in the DANce project.

- the importance of the pleiades, why we restrict to it.

It must be clear what are the limitations, the boundaries in which the objective will be searched

Description of the current methodologies used to address the question mentioned previously.

- The works of Sarro, Krone-Martins, Malo, Gagne etc. LAcweing

-The advantages and caveats of the previous methodologies. 

-It must be clear the necessity of a new perspective

The proposal we made. The use of Bayesian Hierarchical Models. Benefits and issues of BHM.

Description of the advantages of BHM.

-> It must be clear that BHM are the best choice.

Description of the practical issues needed to be solved in order to use BHM.

MCMC techinques and  PSO.

-> It must be clear that MCMC methods are the best option.

Brief descriptions of our results and how they impact our current knowledge.

-> It must be clear that we attained the objective: The pleiades velocity, spatial and mass distributions.
 



