%!TEX root = ../thesis.tex
\chapter{Introduction}
\label{chap:introduction}

\nocite{*} % remove this line to add only cited references

Description of:

- Motivation to investigate nearby young open clusters.

- The importance of the IMF

- Current state of knowledge in dynamical simulations.

- The problematic of constrains in dynamical theories.

It must be clear which is the objective

Description of the Nearby open clusters and their properties.

\section{The DANCe project}

- List of open clusters in the DANce project.

- the importance of the pleiades, why we restrict to it.

It must be clear what are the limitations, the boundaries in which the objective will be searched

Description of the current methodologies used to address the question mentioned previously.

- The works of Sarro, Krone-Martins, Malo, Gagne etc.

-The advantages and caveats of the previous methodologies. 

-It must be clear the necessity of a new perspective

The proposal we made. The use of Bayesian Hierarchical Models. Benefits and issues of BHM.

Description of the advantages of BHM.

-> It must be clear that BHM are the best choice.

Description of the practical issues needed to be solved in order to use BHM.

MCMC techinques and  PSO.

-> It must be clear that MCMC methods are the best option.

Brief descriptions of our results and how they impact our current knowledge.

-> It must be clear that we attained the objective: The pleiades velocity, spatial and mass distributions.
 



