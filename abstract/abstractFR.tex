%!TEX root = ../thesis.tex
{\LARGE French abstract}\\

Il semble maintenant établi que la majorité des étoiles se forment dans des amas \citep{2000AJ....120.3139C, 2003AJ....126.1916P,2003ARA&A..41...57L}. Comprendre l'origine et l'évolution des populations stellaires est donc l'un des plus grands défis de l'astrophysique moderne. Malheureusement, moins d'un dixième de ces amas restent gravitationellement liés au delà de quelques centaines de millions d'années \citep{2003ARA&A..41...57L}. L’étude des amas stellaires doit donc se faire avant leur dissolution dans la galaxie.

Le projet Dynamical Analysis of Nearby Clusters \cite[DANCe,][]{Bouy2013}, dont le travail fait partie, fournit le cadre scientifique pour l'analyse des amas proches et jeunes (NYC) dans le voisinage solaire. Les observations de l'amas ouvert des Pléiades par le projet DANCe offrent une opportunité parfaite pour le développement d'outils statistiques visant à analyser les premières phases de l'évolution des amas.

L'outil statistique développé ici est un système intelligent probabiliste qui effectue une inférence bayésienne des paramètres régissant les fonctions de densité de probabilité (PDF) de la population de l'amas (PDFCP). Il a été testé avec les données photométriques et astrométriques des Pléiades du relevé DANCe. Pour éviter la subjectivité de ces choix des priors, le système intelligent les établit en utilisant l'approche hiérarchique bayésienne (BHM). Dans ce cas, les paramètres de ces distributions, qui sont également déduits des données, proviennent d'autres distributions de manière hiérarchique.

Dans ce système intelligent BHM, les vraies valeurs du PDFCP sont spécifiées par des relations stochastiques et déterministes représentatives de notre connaissance des paramètres physiques de l'amas. Pour effectuer l'inférence paramétrique, la vraisemblance (compte tenu de ces valeurs réelles), tient en compte des propriétés de l'ensemble de données, en particulier son hétéroscédasticité et des objects avec des valeurs manquantes.

Le BHM obtient les PDF postérieures des paramètres dans les PDFCP, en particulier celles des distributions spatiales, de mouvements propres et de luminosité, qui sont les objectifs scientifiques finaux du projet DANCe. Dans le BHM, chaque étoile du catalogue contribue aux PDF des paramètres de l'amas proportionnellement à sa probabilité d'appartenance. Ainsi, les PDFCP sont exempts de biais d'échantillonnage résultant de sélections tronquées au-dessus d'un seuil de probabilité défini plus ou moins arbitrairement.

Comme produit additionnel, le BHM fournit également les PDF de la probabilité d'appartenance à l'amas pour chaque étoile du catalogue d'entrée, qui permettent d'identifier les membres probables de l'amas, et les contaminants probables du champ. La méthode a été testée avec succès sur des ensembles de données synthétiques (avec une aire sous la courbe ROC de 0,99), ce qui a permis d'estimer un taux de contamination pour les PDFCP de seulement 5,8 \%.

Ces nouvelles méthodes permettent d'obtenir et/ou de confirmer des résultats importants sur les propriétés astrophysiques de l'amas des Pléiades. Tout d'abord, le BHM a découvert 200 nouveaux candidats membres, qui représentent 10\% de la population totale de l'amas. Les résultats sont en excellent accord (99,6\% des 100 000 objets dans l'ensemble de données) avec les résultats précédents trouvés dans la littérature, ce qui fournit une validation externe importante de la méthode. Enfin, la distribution de masse des systèmes actuelle (PDSMD) est en général en bon accord avec les résultats précédents de \citet{Bouy2015}, mais présente l'avantage inestimable d'avoir des incertitudes beaucoup plus robustes que celles des méthodes précédentes.

Ainsi, en améliorant la modélisation de l'ensemble de données et en éliminant les restrictions inutiles ou les hypothèses simplificatrices, le nouveau système intelligent, développé et testé dans le présent travail, représente l'état de l'art pour l'analyse statistique des populations de NYC.
%%%%%%%%%%%%%%%%%
