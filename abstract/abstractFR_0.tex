%!TEX root = thesis.tex

%This is the abstract of your PhD thesis.

L'origine et l'évolution des populations stellaires est l'un des plus grands défis de l'astrophysique moderne. On sait que la majorité des étoiles a son origine dans des grappes stellaires \citep{2000AJ....120.3139C, 2003AJ....126.1916P,2003ARA&A..41...57L}. Cependant, seulement moins d'un dixième de ces grappes reste attachée après les premiers cent millions d'années \citep{2003ARA&A..41...57L}. Ergo, la compréhension de l'origine et de l'évolution des étoiles demande des analyses méticuleuses des grappes stellaires à ces âges cruciaux.
Le projet Dynamical Analysis of Nearby Clusters \cite[DANCe,][]{Bouy2013}, dont le présent travail fait partie, fournit le cadre scientifique pour l'analyse des amas proche, ouvert et jeune (NYOC) dans le quartier solaire ($\leq$ 500 pc). Les observations DANCe de l'amas ouvert et bien connu, Pleiades, constituent l'étude de cas parfaite pour le développement et l'essai d'outils statistiques visant à analyser les premières phases de l'évolution des amas.
L'outil statistique développé ici est un système intelligent probabiliste qui effectue une inférence bayésienne pour les paramètres régissant les fonctions de densité de probabilité (PDF) de la population de l'ama (PDFCP). Il a été testé avec les données photométriques et astrométriques de Pléiades du sondage DANCe. Comme tout cadre bayésien, il faut la mise en place de prieurs. Pour éviter la subjectivité de ceux-ci, le système intelligent les établit en utilisant l'approche hiérarchique bayésienne (BHM). Dans ce cas, les paramètres des distributions prior, qui sont également déduits des données, proviennent d'autres distributions de manière hiérarchique.

Dans ce système intelligent BHM, les valeurs vraies du PDFCP sont spécifiées par des relations stochastiques et déterministes représentant l'état de connaissance du NYOC. Pour effectuer l'inférence paramétrique, la probabilité de resemblance des données, compte tenu de ces valeurs réelles, tienne en compte des propriétés de l'ensemble de données, en particulier son hétéroscédasticité et des objects avec des valeur manquantes. En comptabilisant correctement ces propriétés, le système intelligent: i) Augmente la taille de l'ensemble de données, par rapport à des études antérieures travaillant exclusivement sur des objets entièrement observés, et ii) Evite les biais associés aux ensembles de données entièrement observés et les restrictions aux objects de bas incertitude (procédure $\sigma$-clipping).

Le BHM obtiens les PDF postérieurs des paramètres dans les PDFCP, en particulier ces des distributions spatiales, des mouvements propres et de la luminosité. Dans le BHM, chaque objet de l'ensemble de données contribue aux PDF des paramètres proportionnellement à sa probabilité de resemblance. Ainsi, les PDFCP sont exempts de bias résultant de sélections de probabilité typiques d'appartenance (biais d'échantillonnage).

En tant que un sous-produit, le BHM fournit également les PDF de la probabilité d'appartenance au cluster pour chaque objet dans l'ensemble de données. Ces PDFs avec qu'un seuil optimal de classification des probabilités, issus des ensembles de données synthétiques, permettent de classer les objets dans les populations de grappes et de champs. Ce classificateur présente d'excellents résultats, lorsqu'il est appliqué sur des ensembles de données synthétiques (avec une zone sous la courbe ROC de 0,99). À partir de l'analyse des ensembles de données synthétiques, la valeur attendue du taux de contamination pour les PDFCP est de $5,8 \pm 0,2$\%.

Voici les résultats astrophysiques les plus importants de la BHM appliquée à l'ama des Pléiades. Tout d'abord, utilisé comme classificateur, il trouve 200 nouveaux membres candidats, représentant 10\% de nouvelles découvertes. Néanmoins, il montre un accord exceptionnel (99,6\% des 105 objets dans l'ensemble de données) avec les résultats précédents de la littérature. Deuxièmement, la distribution masse des systèmes au présentée (PDSMD) est en général d'accord avec les résultats précédents de \citet{Bouy2015}.

Ainsi, en améliorant la modélisation de l'ensemble de données et en éliminant les restrictions inutiles, le nouveau système intelligent, développé et testé dans le présent travail, représente l'état de l'arte pour l'analyse statistique des populations de NYOC.

