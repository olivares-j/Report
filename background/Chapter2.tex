%!TEX root = ../thesis.tex
\chapter{The Pleiades as a benchmark}
\label{chap:pleiades}

\section{Generalities}
The ancient greeks named Pleiades to a group of stars which they believe had a common origin. These stars were the seven sisters, which, together with their parents the titan Atlas and the nymph Pleione, were put in the sky  by the god Zeus.
 
Today we call the Pleiades cluster not just to the nine stars that made up the original Pleione family but to a much larger group, which according to \citet{Bouy2015} goes up to $\sim2100$ members. This cluster is fairly close to the sun, $\sim 136$ pc according to \citet{Galli2017}, and is also young in galactic scales, with only $\sim120$ Myr \citep{Stauffer1998}. Since it is located in the solar neighbourhood it has a distinctive velocity, when compared to that of the far distant objects, of about $-16 mas/yr$ in right ascension and $20 mas/yr$ in declination. Also, it has expelled most of its cocoon gas, which gives it an almost null extinction of $A_v=0.12$ \citep{Guthrie1987}. 

The previous properties make the pleiades the most studied cluster in the history of astronomy. In the following sections I will describe with more detail the previous properties and give others relevant for the present work.

\section{The distance to the Pleiades}

\subsection{Measuring distances}
In astronomy, measuring distances is a complicated task. Techniques vary according to the distance scale that we aim to measure. The distance ladder is constructed from smaller to larger distances. The first step in that ladder is the distance to the sun. After that, the distance to the planets and then to the stars. Since this works deal only with nearby clusters, I only explain the measuring distance to these objects. 

The most direct way to measure distance to nearby stars is by means of the trigonometric parallax. This is the relative angular displacement, with respect to the far distant stars, that an object suffers in the course of a year. This relative displacement is time dependent and results from the movement of the earth (thus the observer) on its orbit around the sun. The relative displacement is maximal when measurements are taken at diametrically opposed points in the earth orbit, thus when they are separated by six months. This maximal displacement is called the parallax of the object. The distance to the object is then obtained by inverting the angular distance, measured in seconds of arc. By doing so, we obtain the distance measured in parsecs. This measurement unit gets its name from parallax-second. Thus an objects at distance one parsec from the sun shows a parallax of one arc second. The further the object is, the smaller the parallax. 

As any other measurement, parallaxes had uncertainties. This uncertainties usually are a proxy for the width of the parallax distribution. Since parallaxes are related to the inverse of the distance, then the vast majority of stars had parallaxes near zero. Then, given certain precision of an instrument, and a distant object, nothing prohibits that this object may have negative measurements of its parallax. The parallax distribution is a non limited continuous distribution. 

When transforming parallaxes into distances we may be tempted to take an statistics of the distribution, the mean for example, and just invert it to obtain the distance. Since this is the definition it will hold only if we have the true distance. The true distance is that in which the uncertainties are negligible. However, because measurements always have uncertainties, the inversion of the parallax will render an unbiased estimate of the true distance only for small values of the relative uncertainty \cite[][mention that a reasonable value is below 0.15-0.20]{Lutz1973}. The shape of the parallax distribution plays an important roll. If we are interested in the distance and we only have the parallax distribution, this distribution must be transformed into that of distances. However, this transformation is not a simple inversion.  
Several authors have proposed different approaches to the problem of distance determination using parallaxes, see for example \citet{Lutz1973,2015PASP..127..994B,2016ApJ...832..137A,2016ApJ...833..119A}. The proper way, as \citet{2015PASP..127..994B} points out is to infer the true distances given the observed parallaxes. For that, a prior on the distance must be established. The afore mentioned authors describe three different kinds of priors and the methodology needed to infer the true distances.

Now, I focus on the particular case of the distance to the Pleiades. The first parallax measurement of the Pleiades distance was done by \citet{1999A&A...341L..71V} using the \emph{Hipparcos} data. Later himself \citep{2009A&A...497..209V} refined its sample and obtained a value of $120\pm1.9pc$. However, \citet{2000ApJ...533..938G, 2005AJ....129.1616S} using also the parallaxes of smaller samples (seven and three, respectively) of stars, measured values of $130.9\pm7.4pc$ and $134.6\pm3.1pc$, respectively. Finally, \citet{2014Sci...345.1029M} measured $136.2\pm1.2pc$ using parallaxes of three stars. There is a clear controversy between \emph{Hipparcos} data and that of the rest of the parallax measurements. This controversy will be probably solved by \emph{Gaia}. 

Until this controversy have been solved, I have decided to choose the distance found by our research group, $134.4^{2.9}_{2.8}pc$ \citep{Galli2017}. We found this distance using the kinematic parallaxes delivered by the moving cluster technique. This essentially exploits the fact that since clusters are bound, their members show a clear kinematic footprint: they seem to converge to a point in the sky \citep{1964IAUS...20...50B}. Using this point and the velocity of the members (proper motion and radial velocities) it is possible to derive individual parallaxes. Furthermore, these individual parallaxes show a distribution which results from the dispersion of the cluster members along the line of sight (XXXXXXXCheck if phillips plots could be used XXXXXXX). However, this distribution is only the depth component of the space distribution of the cluster, the other two components are given by the spatial distribution. 

\section{Spatial Distribution}
The spatial distribution, as I mentioned, is the two dimensional projection of the space distribution of the cluster in the plane perpendicular to the line of sight. In general, individual object positions in the plane of the sky, commonly known as the coordinates right ascension (R.A.) and declination (Dec.), are more easily measured than individual parallaxes. For this reason, just a small fraction of objects has parallaxes. In the case of the Pleiades, only $\sim70$ members out of $\sim2100$ have \emph{Hipparcos}'s parallaxes. I found this figure after cross matching the Tycho-2 candidate members of \citet{Bouy2015} with the \emph{Hipparcos} catalogue \citep{1997A&A...323L..49P} . In addition, the relative uncertainties in R.A. and Dec. coordinates are far better ($10^{-5}$) than those of individual parallaxes ($10^{-1}$).

Due to the previous considerations, the space distribution of the Pleiades has been studied mainly trough its spatial distribution.
It has been the subjects of several studies. One of the earliest results of the Pleiades spatial distribution was done by \citet{Limber1962} who fitted the spatial distribution of the 246 candidate members of \citet{Trumpler1921}. These members were contained in a $3^o$ radius around \emph{Alcyone} (one of the central most massive stars of the cluster). He used a mixture of four indices polytropic distributions as described is his earlier \citet{Limber1961} work. Later, \citet{Pinfield1998} fitted King profiles \citep{King1962} of different masses ($5.2,1.65,0.83,0.3 \ \ M_{\odot}$) to candidate members from the literature extending until a $3^o$ radius. They estimated a tidal radius of $13.1pc$, in which $1194$ members were contained. These amounted to a total mass of $735\ \ M_{\odot}$. The mean mass they estimated is $0.616\ \ M_{\odot}$. On the same year \citet{Raboud1998} fitted a King's profile to a list of 270 candidate members contained within a $5^o$ radius. They found a core radius of $1.5\ \ pc$ and a tidal radius of $17.5\ \ pc$. Using different approaches their derived total mass is in the range $500$ to $8000 \ \ M_{\odot}$. They also measured an ellipticity of $0.17$, however they do not made any explicit mention on the position angle of the axis of the ellipse.

Later, \citet{Adams2001} also fitted a King's profile to a list of $\sim 4233$ objects within a radius of $10^o$. These objects had membership probabilities, $p>0.1$. They found a core radius of $2.35-3.0\ \ pc$ and a tidal radius of $13.6-16\ \ pc$. They estimate a total mass of $\sim 800 \ \ M_{\odot}$, and ellipticities in the range $0.1-0.35$. \citet{Converse2008} used a sample of 1245 from \citet{Stauffer2007} to fit a King's profile. They obtained a tidal radius of $18\ \ pc$ and a core radius of  $1.3 \ \ pc$. Later, \citet{Converse2010} refined their study and obtained a core radius of $2.0\pm0.1 \ \ pc$, a tidal radius of $19.5 \pm 1.0 \ \ pc $ and a total mass of $870\pm35 M_{\odot}$.

The previous summary of results shows at least two interesting points. The first one, is that the King's profile \citep{King1962} has bee the preferred choice. Although this profile has its origins on the globular clusters domain, in which the end of the cluster is well determined. It has been also widely applied to open clusters, particularly the pleiades one. The second one concern the trend of the tidal radius as a function of year of publication and size of the survey. Except the work of \citet{Adams2001} in which the sample is large and contains low membership probability objects. Although these author mention that their high membership probability objects ($1200$) are contained in a $6^o$ radius. These two aspects are bonded. The King's tidal radius will continue to increase until the physical end of the cluster will be within the survey's radius, just as it happens with the globular clusters. 
  
\section{Velocity Distribution}
\subsection{Radial Velocity}

\section{Luminosity Distribution}

\section{Mass Distribution}


\subsection{Total Mass of the cluster}
\citet{Limber1961} $760-900 M_{\odot}$
\citet{Pinfield1998}
\section{The current dynamical scenario}
\subsection{Pleiades time-scales}
Pinfield equation 13 and 14

\section{The Pleiades DANCe DR2.}
\label{sect:DR2}
This section must contain a detailed description of the DR2 data.

\begin{table}[htdp]
\caption{Pleiades DANCe DR2 properties}
\begin{center}
\begin{tabular}{|c|c|}

\end{tabular}
\end{center}
\label{tab:DR2properties}
\end{table}%
\subsection{Particularities of the Pleiades DANCe DR2}
As described in Section \ref{Chapter2:Sect:DR2} the Pleiades DANCe DR2 contains astrometric (stellar positions and proper motions) and photometric ($ugrizYJHK_s$) measurements for 1,972,245 objects. 



\subsection{Selection of variables}
\citet{Sarro2014} demonstrated that the most effective variables for the discrimination of members are the proper motions and the $riYJHK_s$ bands. However, they excluded the $r$ band due to its large number of missing values in their training set. The selection of variables in this work aims at comparing its results with those found by \citet{Bouy2015} using the $\mu_{\alpha},\mu_{\delta},J,H,K_s,i-K_s,Y-J$ variables. 

The set of variables used in this work are the stellar positions, the proper motions in right ascension and declination, $\mu_{\alpha},\mu_{\delta}$, and the photometric colours and magnitudes, $i-K_s$,$Y,J,H,K_s$. However, in order to compare the results with those of \citet{Bouy2015}, the analysis of the spatial distribution of the Pleiades stellar positions is done independently, in \citet{Olivares2017b}.

As described in \citet{Olivares2017}, the photometry is modelled by parametric series of cubic spline. The parameter of this series is the colour index $i-K_s$ (in the following $CI$). This colour allows the most one-to-one variate-covariate relation. Figure \ref{fig:otherCI} shows the colour-magnitude diagram (CMD) $K$ vs $Y-J$ the second most one-to-one relation.


\begin{figure}[htbp]
\begin{center}
%\includegraphics[width=\textwidth]{}
\caption{$K$ vs $Y-J$ CMD for the Pleiades candidate members of \citet{Bouy2015}}.
\label{fig:otherCI}
\end{center}
\end{figure}

\subsection{Data preprocessing}
 Since both photometry and proper motions carry crucial information for the disentanglement of the cluster population, we restrict the data set to objects with proper motions and at least two observed values in any of our four CMDs: $Y,J,H,K_s$ vs $CI$. This restriction excludes 22 candidate members of \citet{Bouy2015}, which have only one observed value in the photometry. Furthermore, we restrict the lower limit ($CI =0.8$) of the colour index to the value of the brightest cluster member. We do not expect to find new bluer members in the bright part of the CMDs. We set the upper limit ($CI=8$) of the colour index at one magnitude above the colour index of the reddest known cluster member, thus allowing for new discoveries. Due to the sensitivity limits of the DR2 survey in $i$ and $K_s$ bands, objects with $CI>8$ have $K_s$ magnitudes $\geq 16$ mag. These objects are incompatible with the cluster sequence and therefore we discard them a priori as cluster members.

Our current computational constraints and the costly computations associated to our methodology (described throughout this Sect.), prevent its application to the entire data set. However, since the precision of our methodology, as that of any statistical analysis, increases with the number of independent observations, we find that a size of $10^5$ source for our data is a reasonable compromise. Although a smaller data set produces faster results, it also renders a less precise model of the field (in the area around the cluster) and therefore, a more contaminated model of the cluster. For these reasons, we restrict our data set to the $10^5$ objects with highest membership probabilities according to \citet{Bouy2015}. Of this resulting data set, the majority ($\approx$98\%) are field objects with cluster membership probabilities around zero. Thus, the probability of leaving out a cluster member is negligible. For the remaining of the objects in the Pleiades DANCe DR2, we assign membership probabilities \emph{a posteriori}, once the cluster model is constructed.



