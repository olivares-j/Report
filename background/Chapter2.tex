%!TEX root = ../thesis.tex
\chapter{The Pleiades as a benchmark}
\label{chap:pleiades}

\section{Generalities}
The ancient greeks named Pleiades to a group of stars which they believe had a common origin. These stars were the seven sisters, which, together with their parents the titan Atlas and the nymph Pleione, were put in the sky  by the god Zeus.
 
Today we call the Pleiades cluster not just to the nine stars that made up the original Pleione family but to a much larger group, which according to \citet{Bouy2015} goes up to $\sim2100$ members. This cluster is fairly close to the sun, $\sim 136$ pc according to \citet{Galli2017}, and is also young in galactic scales, with only $\sim120$ Myr \citep{Stauffer1998}. Since it is located in the solar neighbourhood it has a distinctive velocity, when compared to that of the far distant objects, of about $-16 mas/yr$ in right ascension and $20 mas/yr$ in declination. Also, it has expelled most of its cocoon gas, which gives it an almost null extinction of $A_v=0.12$ \citep{Guthrie1987}. 

The previous properties make the pleiades the most studied cluster in the history of astronomy. In the following sections I will describe the current knowledge on the most relevant astrophysical quantities for this work: the distance, positions, velocities, luminosities and mass of the cluster. The statistical distributions of these properties comprise the objective of the present work. I will focus primarily on the statistical aspects, however, if any I will also give notice on the modelling of these distributions.
 
\section{The distance to the Pleiades}

\subsection{Measuring distances}
In astronomy, measuring distances is a complicated task. Techniques vary according to the distance scale that we aim to measure. The distance ladder is constructed from smaller to larger distances. The first step in that ladder is the distance to the sun. After that, the distance to the planets and then to the stars. Since this works deal only with nearby clusters, I only explain the measuring distance to these objects. 

The most direct way to measure distance to nearby stars is by means of the trigonometric parallax. This is the relative angular displacement, with respect to the far distant stars, that an object suffers in the course of a year. This relative displacement is time dependent and results from the movement of the earth (thus the observer) on its orbit around the sun. The relative displacement is maximal when measurements are taken at diametrically opposed points in the earth orbit, thus when they are separated by six months. This maximal displacement is called the parallax of the object. The distance to the object is then obtained by inverting the angular distance, measured in seconds of arc. By doing so, we obtain the distance measured in parsecs. This measurement unit gets its name from parallax-second. Thus an objects at distance one parsec from the sun shows a parallax of one arc second. The further the object is, the smaller the parallax. 

As any other measurement, parallaxes had uncertainties. This uncertainties usually are a proxy for the width of the parallax distribution. Since parallaxes are related to the inverse of the distance, then the vast majority of stars had parallaxes near zero. Then, given certain precision of an instrument, and a distant object, nothing prohibits that this object may have negative measurements of its parallax. The parallax distribution is a non limited continuous distribution. 

When transforming parallaxes into distances we may be tempted to take an statistics of the distribution, the mean for example, and just invert it to obtain the distance. Since this is the definition it will hold only if we have the true distance. The true distance is that in which the uncertainties are negligible. However, because measurements always have uncertainties, the inversion of the parallax will render an unbiased estimate of the true distance only for small values of the relative uncertainty \cite[][mention that a reasonable value is below 0.15-0.20]{Lutz1973}. The shape of the parallax distribution plays an important roll. If we are interested in the distance and we only have the parallax distribution, this distribution must be transformed into that of distances. However, this transformation is not a simple inversion.  
Several authors have proposed different approaches to the problem of distance determination using parallaxes, see for example \citet{Lutz1973,2015PASP..127..994B,2016ApJ...832..137A,2016ApJ...833..119A}. The proper way, as \citet{2015PASP..127..994B} points out is to infer the true distances given the observed parallaxes. For that, a prior on the distance must be established. The afore mentioned authors describe three different kinds of priors and the methodology needed to infer the true distances.

Now, I focus on the particular case of the distance to the Pleiades. The first parallax measurement of the Pleiades distance was done by \citet{1999A&A...341L..71V} using the \emph{Hipparcos} data. Later himself \citep{2009A&A...497..209V} refined its sample and obtained a value of $120\pm1.9pc$. However, \citet{2000ApJ...533..938G, 2005AJ....129.1616S} using also the parallaxes of smaller samples (seven and three, respectively) of stars, measured values of $130.9\pm7.4pc$ and $134.6\pm3.1pc$, respectively. Finally, \citet{2014Sci...345.1029M} measured $136.2\pm1.2pc$ using parallaxes of three stars. There is a clear controversy between \emph{Hipparcos} data and that of the rest of the parallax measurements. The current \emph{Tycho-Gaia} data release, TGAS, shows that the Pleiades parallax is $\pi = 7.48\pm0.03\ \ mas$ \citep{2017A&A...601A..19G}. This seems to indicate that the \emph{Hipparcos} parallaxes were somehow biased. However, this controversy will remain until the totally independent \emph{Gaia} DR2.

Until this controversy have been solved, I have decided to choose the distance found by our research group, $134.4^{+2.9}_{-2.8}pc$ ($7.44\pm0.08\ \ mas$) \citep{Galli2017}, which is in good agreement with the one found by TGAS. We found this distance using the kinematic parallaxes delivered by the moving cluster technique. This essentially exploits the fact that since clusters are bound, their members show a clear kinematic footprint: they seem to converge to a point in the sky \citep{1964IAUS...20...50B}. Using this point and the velocity of the members (proper motion and radial velocities) it is possible to derive individual parallaxes. Furthermore, these individual parallaxes show a distribution which results from the dispersion of the cluster members along the line of sight. Figures \ref{fig:parallaxPhillip} and \ref{fig:parallaxTGAS} show the distribution of parallaxes for the Pleiades candidate members according to \citet{Galli2017} and \citet{2017A&A...601A..19G}. However, this distribution is only the depth component of the space distribution of the cluster, the other two components are given by the spatial distribution. 

\begin{figure}[htbp]
\begin{center}
%\includegraphics[width=\textwidth]{}
\caption{Distribution of parallax for the Pleiades according to \citet{Galli2017}}
\label{fig:parallaxPhillip}
\end{center}
\end{figure}

\begin{figure}[htbp]
\begin{center}
%\includegraphics[width=\textwidth]{}
\caption{Distribution of parallax for the Pleiades according to \citet{2017A&A...601A..19G}}

\label{fig:parallaxTGAS}
\end{center}
\end{figure}


\section{Spatial Distribution}
The spatial distribution, as I mentioned, is the two dimensional projection of the space distribution of the cluster in the plane perpendicular to the line of sight. In general, individual object positions in the plane of the sky, commonly known as the coordinates right ascension (R.A.) and declination (Dec.), are more easily measured than individual parallaxes. For this reason, just a small fraction of objects has parallaxes. In the case of the Pleiades, only $\sim70$ members out of $\sim2100$ have \emph{Hipparcos}'s parallaxes. I found this figure after cross matching the Tycho-2 candidate members of \citet{Bouy2015} with the \emph{Hipparcos} catalogue \citep{1997A&A...323L..49P} . In addition, the relative uncertainties in R.A. and Dec. coordinates are far better ($10^{-5}$) than those of individual parallaxes ($10^{-1}$).

Due to the previous considerations, the space distribution of the Pleiades has been studied mainly trough its spatial distribution.
It has been the subjects of several studies. One of the earliest results of the Pleiades spatial distribution was done by \citet{Limber1962} who fitted the spatial distribution of the 246 candidate members of \citet{Trumpler1921}. These members were contained in a $3^o$ radius around \emph{Alcyone} (one of the central most massive stars of the cluster). He used a mixture of four indices polytropic distributions as described is his earlier \citet{Limber1961} work. Later, \citet{Pinfield1998} fitted King profiles \citep{King1962} of different masses ($5.2,1.65,0.83,0.3 \ \ M_{\odot}$) to candidate members from the literature extending until a $3^o$ radius. They estimated a tidal radius of $13.1pc$, in which $1194$ members were contained. These amounted to a total mass of $735\ \ M_{\odot}$. The mean mass they estimated is $0.616\ \ M_{\odot}$. On the same year \citet{Raboud1998} fitted a King's profile to a list of 270 candidate members contained within a $5^o$ radius. They found a core radius of $1.5\ \ pc$ and a tidal radius of $17.5\ \ pc$. Using different approaches their derived total mass is in the range $500$ to $8000 \ \ M_{\odot}$. They also measured an ellipticity of $0.17$, however they do not made any explicit mention on the position angle of the axis of the ellipse.

Later, \citet{Adams2001} also fitted a King's profile to a list of $\sim 4233$ objects within a radius of $10^o$. These objects had membership probabilities, $p>0.1$. They found a core radius of $2.35-3.0\ \ pc$ and a tidal radius of $13.6-16\ \ pc$. They estimate a total mass of $\sim 800 \ \ M_{\odot}$, and ellipticities in the range $0.1-0.35$. \citet{Converse2008} used a sample of 1245 from \citet{Stauffer2007} to fit a King's profile. They obtained a tidal radius of $18\ \ pc$ and a core radius of  $1.3 \ \ pc$. Later, \citet{Converse2010} refined their study and obtained a core radius of $2.0\pm0.1 \ \ pc$, a tidal radius of $19.5 \pm 1.0 \ \ pc $ and a total mass of $870\pm35 M_{\odot}$.

The previous summary of results shows at least two interesting points. The first one, is that the King's profile \citep{King1962} has bee the preferred choice. Although this profile has its origins on the globular clusters domain, in which the end of the cluster is well determined. It has been also widely applied to open clusters, particularly the pleiades one. The second one concern the trend of the tidal radius as a function of year of publication and size of the survey. Except the work of \citet{Adams2001} in which the sample is large and contains low membership probability objects. Although these author mention that their high membership probability objects ($1200$) are contained in a $6^o$ radius. These two aspects are bonded. The King's tidal radius will continue to increase until the physical end of the cluster will be within the survey's radius, just as it happens with the globular clusters. 
  
\section{Velocity Distribution}

The three dimensional velocity distribution has been also studied using its projections, one along the line of sight, which corresponds to the radial velocity, and another one in the plane of the sky, which corresponds to the proper motions. These later ones are angular velocities obtained after measuring the angular displacement that an object shows when measured  in at least in two different epochs. Again, measuring the stellar positions and its displacement on time is easier to obtain than the measuring of the radial velocities. These last ones are estimated through the Doppler shifting of the absorption lines in spectre of the object. Although more precise (usually on the $1 \ \ km\cdot s^{-1}$ regime) radial velocities demand large telescopes and longer observing times. For these reasons, historically the velocity distribution has been studied through the spatial velocity or proper motions distribution. 

Probably the first description of the spatial velocity distribution of the Pleiades is that of \citet{1884MNRAS..44..355P}. Using archival data from  Königsberg (1838-1841), Paris(1874) and Oxford (1878-1880), together with his own \emph{Differential Micrometer} observations, he was able to observe the relative displacements of 40 pleiades stars. According to him \citep{1884MNRAS..44..355P}: \textit{the relative displacements of these distant suns, although not distinctly and accurately measurable in numerical extent, appear to vary both in direction and amount; indicating thereby the mutual influence of a group of gravitating bodies.} Later, proper motion measurements continues to be done until \citet{Trumpler1921} use them to find the members of the cluster. He classified objects as members according to the distance they show to the mean proper motion of the cluster.  This mean was calculated by Boss in his \emph{Preliminary General Catalog} \citep{Trumpler1921}. So far as my historic research went, this was the first measurement of an statistic of the spatial velocity distribution of the Pleiades. Later \citet{1938AJ.....47...25T} using Trumpler's data an archival compilations was able to measure an internal dispersion in the proper motions of $0.79\ \ mas\cdot yr^{-1}$. From this value he derived a mass of $260\ \ M_{\odot}$. Probably this was the first measurement of the second moment of the spatial velocity distribution.

In recent years \citet{Pinfield1998} used the velocity dispersion of the cluster to probe that it is in near virial equilibrium. Later, \citet{2006ARep...50..714L} used the projected radial and tangential velocity components of the spatial velocity distribution of 340 members to claim the absence of evidence for rotation, expansion or compression of the cluster. Also, he does not found evidence of mass segregation in the spatial distribution. Finally, in \citet{Galli2017} we found a velocity dispersion of $0.93 \ \ km\cdot s^{-1}$, and the projected velocity distribution in the direction perpendicular to the great circle that joins the star with the convergent point of the cluster. This distribution (shown on Figure 10 of the mentioned work) has a dispersion of $1.45 \ \ mas\cdot yr^{-1}$. 

Concerning the radial velocities, the first record in the Pleiades is that of \citet{1904ApJ....19..338A}. He measured the radial velocities of six of the most brightest stars of the cluster. The latest one is the compilation of literature measurements made by \citet{Galli2017}. This list contains measurements for 394 objects. This distribution is centred at $5.6\ \ km \cdot s^{-1}$ and is almost gaussian (see Fig. 5 of the mentioned work). 

As I will explain later, the complete velocity distribution is a key ingredient in the understanding of the cluster dynamics. Although, spatial and radial velocities are useful projections of the complete distribution, the dynamical analysis of the cluster demands the complete velocity distribution. In \citet{Galli2017} we provide a list of 64 cluster members with full spatial velocities. 
\section{Luminosity Distribution}

The study of the distribution of luminosities in the Pleiades started few years later than that of the positions and proper motions. The first record I found on the luminosity distribution is the one of \citet{Trumpler1921} (see Fig. \ref{fig:luminosityTrumpler}). He computed the number of stars in each magnitude bin for his two samples of candidate members, those comprising the objects within the central $1^o$, and those within $1^o$ and $3^o$, referred as Tables I and II, respectively. The completeness of inner sample was estimated at 14.5 magnitudes whereas that of the outer at  9.0 magnitudes, both for the visual band. He observed that the luminosity distributions of these two samples were not alike. The inner sample is brighter than the outer one. He also observed that the luminosity distribution is not smooth and shows a local minimum at $9-10$ magnitudes, then an abrupt rise. Both effects in the two samples.

\begin{figure}[htbp]
\begin{center}
%\includegraphics[width=\textwidth]{}
\caption{Luminosity distribution of \citet{Trumpler1921}}
\label{fig:luminosityTrumpler}
\end{center}
\end{figure}

Later, \citet{Johnson1958} obtained the luminosity distribution using a sample of 289 candidate members. They assessed  membership solely on photometry. Their luminosity distribution is shown in Fig. \ref{fig:luminosityJohnson}
Later, \citet{Limber1962} compare the luminosity functions derived from the data of \citet{Trumpler1921}, \citet{Hertzsprung1947}, and \citet{Johnson1958}, see Fig. \ref{fig:luminosityLimber}. These last distributions are complete until visual magnitudes of $8.5$ and $9.5$ mag. He also compared them with the initial luminosity distribution, and the present day luminosity distribution of the solar neighbourhood. He notes that the differences between observed and predicted luminosities start to happen at absolute magnitude $5.5$.

\begin{figure}[htbp]
\begin{center}
%\includegraphics[width=\textwidth]{}
\caption{Luminosity distribution in the visual band according to \citet{Johnson1958}.}
\label{fig:luminosityJohnson}
\end{center}
\end{figure}


\begin{figure}[htbp]
\begin{center}
%\includegraphics[width=\textwidth]{}
\caption{Luminosity distribution in the visual band according to \citet{Limber1962}.}
\label{fig:luminosityLimber}
\end{center}
\end{figure}

In recent years, the luminosity distribution has been described in the works of \citet{Lodieu2012} and \citet{Bouy2015}. 
\citet{Lodieu2012} using the \emph{UKIDSS} DR9 survey for galactic clusters and a probabilistic members selection method (see discussion in Chapter \ref{chap:introduction}) based on proper motions, and proper motions and photometry, found 8797 and 1147 candidate members, respectively. However, they do not provide the contamination rate in their analysis. Using both lists they provide their luminosity distributions in the $Z$ band, which I show in Fig. \ref{fig:luminosityLodieu}.

\begin{figure}[htbp]
\begin{center}
%\includegraphics[width=\textwidth]{}
\caption{Luminosity distribution according to \citet{Lodieu2012}.}
\label{fig:luminosityLodieu}
\end{center}
\end{figure}

In \citet{Bouy2015},  we estimated the present day system luminosity distribution of 1378 candidate members contained within the central $3^o$ region (with the centre at $RA=03:46:48$ and $Dec=24:10:17$ J2000.0). It is called systemic because it has not been corrected for unresolved systems. An unresolved system is a group of stars (e.g. binaries) that due to its compactness appears as a single object. This distribution was computed for the $K_s$ band and is sensitive up to $K_s \sim 20 \ \ mag$ and complete until $K_s \sim 17\ \ mag$. This luminosity distribution is shown in Fig. \ref{fig:luminosityBouy}


\begin{figure}[htbp]
\begin{center}
%\includegraphics[width=\textwidth]{}
\caption{Luminosity distribution in the $K_s$ band according to \citet{Bouy2015}.}
\label{fig:luminosityBouy}
\end{center}
\end{figure}

\section{Mass Distribution}

In astrophysics in general, the mass distribution is a corner stone in the understanding of the star formation process and later evolution of stellar systems. Although the temporal evolution of these systems is mainly dominated by the gravitational potential, the initial conditions and an ongoing star formation process (if any), also contribute to the shape of the mass distribution. This distribution contains the fingerprints of past events of the cluster and plays a key roll in its future evolution. Indeed, it is essential to one of the objectives of modern astrophysics: the determination of the roll played by the initial conditions or the environment, in the temporal evolution of the stellar systems. 

The mass distribution of the Pleiades has been largely studied. Again, the first work on the mass distribution is that of \citet{Limber1962}. Although he did not shows any graphical or tabular representation of it, he gave the luminosity distribution and the mass-luminosity ratio. Form these the mass distributions can be derived. Instead, he use them to obtain the total mass of the cluster ($760 M_{\odot}$,see next section). 

Most probably, the first work to present the mass distribution derived from luminosity distributions and a mass-luminosity relation from theoretical models was that of \citet{Hambly1991}. Using $R$ and $I$ observations from the \emph{United Kingdom Schmidt Telescope Unit} together with the mass-luminosity relation from theoretical isochrone models of Padova group, he was able to transform his luminosity distribution into a mass distribution, see Fig. \ref{fig:massHambly}. 

\begin{figure}[htbp]
\begin{center}
%\includegraphics[width=\textwidth]{}
\caption{Mass distribution of \citet{Hambly1991} derived from luminosity distribution and theoretical isochron models.}
\label{fig:massHambly}
\end{center}
\end{figure}

The mass distribution studies from recent years are again those of \citet{Lodieu2012} and \citet{Bouy2015}. These are shown in Fig. \ref{fig:massLodieu} and \ref{fig:massBouy}. In both the luminosity distributions are transformed into mass distributions using theoretical isochrone models. \citet{Lodieu2012} used a distance of $120.2 \ \ pc$, an age of $120\ \ Myr$, and the \emph{NEXTGEN} theoretical models \citet{1998A&A...337..403B} to derive their mass distribution. \citet{Bouy2015} we use a distance of $136.2\ \ pc$ an age of $120\ \ Myr$ and the \emph{BT-Settl} theoretical isochrone models of \citet{2014IAUS..299..271A}. 


Both works found contrasting aspects in their discussions. In one hand \citet{Lodieu2012} found that their present day mass distribution agrees with previous studies from the literature, and is also consistent with the system field mass function of \citet{Chabrier2005}. \citet{Chabrier2005} fitted a log-normal function to the visual luminosity distribution of the closest $8 \ \ pc$ field objects. On the other hand, in \citet{Bouy2015} we found that although the \citet{Chabrier2005} mass function match that of the Pleiades in the $0.02-0.6\ \ M_{\odot}$ mass range, it predicts to many low-mass stars and brown dwarfs. 

The difference between both Pleiades present day mass distributions could arise from the different samples of members, the different theoretical isochron models, or from both of them. The different distance values used in these two works can not account for the observed differences since they introduce only a general shift in luminosity. 

Concerning the differences between the two isochrone models, in \citet{2013MmSAI..84.1053A} the authors show there are clear differences between the effective temperatures delivered by the \emph{BT-Settl} and the \emph{NEXTGEN} model in the low-mass regime at $5 \ \ Gyr$.  

Concerning the differences between the lists of candidate members, in one hand \citet{Lodieu2012} do not provide (at least explicitly) any estimate of contamination rate of their samples. Furthermore, their membership methodology has some draw-backs \cite[see][]{Sarro2014} that may have biased their results. Therefore, the agreement they found between their present day mass distribution and the one of \citet{Chabrier2005}, which models the field mass distribution, seem to indicate at least the following options: i) the Pleiades present day mass distribution indeed follows that of the field, or ii) their samples are contaminated.  

On the other hand, in \citet{Bouy2015} we estimated a contamination rate of 7\% for which we do not have evidence of being non-homogeneous. Even if this contamination were not homogeneous, its percentage would not be able to account for the observed discrepancies between the \citet{Chabrier2005} mass function and our present day mass distribution. These go up to $30-40\%$ in the low-mass regime.

The previous studies show that there is still work to do in the analysis of the Pleiades mass distribution, particularly at the low-mass range where the theoretical models, both of mass function and isochrones, led to discrepancies in the present day mass distribution.  

\begin{figure}[htbp]
\begin{center}
%\includegraphics[width=\textwidth]{}
\caption{Pleiades present day mass distribution  from \citet{Lodieu2012}}
\label{fig:massLodieu}
\end{center}
\end{figure}

\begin{figure}[htbp]
\begin{center}
%\includegraphics[width=\textwidth]{}
\caption{Pleiades present day mass distribution  from \citet{Bouy2015}}
\label{fig:massBouy}
\end{center}
\end{figure}

\subsection{Total mass of the cluster}
Before ending this section I present a summary of the studies that provided an estimate of the total mass of the cluster.

The first record of the cluster total mass is that of \citet{1938AJ.....47...25T}. Assuming virial equilibrium he estimated it to be $260 \ \ M_{\odot}$. He also computed $200 \ \ M_{\odot}$ using the Eddignton's mass-luminosity relation for objects brighter than $15 \ \ mag$ in the visual band.

Later, the following works continue to report higher masses. \citet{1956MNRAS.116..296W} estimated a total mass of $337 \ \ M_{\odot}$ using Hertzsprung's catalogue. \citet{Limber1962} computed the total mass in two ways. In the first one he assumed the cluster was virialised and obtained a mass of $900 \ \ M_{\odot}$. Using the luminosity function he estimated the lower limit to the total mass in $760 \ \ M_{\odot}$. \citet{1970AJ.....75..563J} measured $470\ \ M_{\odot}$ and $690\ \ M_{\odot}$ using the luminosity distribution and the virial theorem, respectively. \citet{1980IAUS...85..157V} using the virial theorem, a mean mass of $2\ \ M_{\odot}$, and a velocity dispersion of $0.7\ \ km \cdot s^{-1}$ in each spatial direction, determined a total mass of $2000 \ \ M_{\odot}$. \citet{1995JKAS...28...45L} measured $700 \ \ M_{\odot}$ using the luminosity distribution and a mass-luminosity relation. \citet{Pinfield1998} fitting a King's profile to the spatial distribution of the cluster members obtained $735\ \ M_{\odot}$. \citet{2001AJ....121.2053A} counting individual masses of candidate members within $5.5^o$ obtained a total mass of $690 \ \ M_{\odot}$. \citet{Converse2008} found $820 \ \ M_{\odot}$ after adding the individual masses of 1245  candidate members of \citet{Stauffer2007}. To obtain these masses they  transformed the $K$ and $I-K$ magnitude and colour into masses using the mass-luminosity relation given by the theoretical isochrone models of \citet{1998A&A...337..403B}. Later, \citet{Converse2010} they redo their analysis and found the total mass to be $870\pm35\ \ M_{\odot}$.

%\section{The current dynamical scenario}
%The most important sources of gravitational interactions affecting individual objects are due to: other individual objects (like for example close encounters or binary interactions), the ensemble of individual objects (the potential of the cluster itself or its momentum), other ensembles of objects (interactions with other clusters or molecular clouds) and, the galactic potential (perturbations due to the disk, resonances, arms).
%
%\subsection{Pleiades time-scales}
%Pinfield equation 13 and 14

\section{The Pleiades DANCe DR2}
\label{sect:DR2}

The Pleiades DANCe DR2 contains astrometric (stellar positions and proper motions) and photometric ($ugrizYJHK_s$) measurements for 1,972,245 objects. In Table \ref{tab:DR2properties} I provide the basic statistics for these data. Also, to complement this information, in Figs. \ref{fig:pmuncert} to \ref{fig:maguncert} I give the uncertainties of the data set. Table \ref{tab:DR2Missing} contains the number of entries in the DANCe DR2 data set that have missing entries.

\begin{table}[htdp]
\caption{Pleiades DANCe DR2 properties}
\begin{center}
\begin{tabular}{|c|c|}

\end{tabular}
\end{center}
\label{tab:DR2properties}
\end{table}%

\begin{figure}[htbp]
\begin{center}
%\includegraphics[width=\textwidth]{}
\caption{default}
\label{fig:pmuncert}
\end{center}
\end{figure}

\begin{figure}[htbp]
\begin{center}
%\includegraphics[width=\textwidth]{}
\caption{default}
\label{fig:maguncert}
\end{center}
\end{figure}

\begin{table}[htdp]
\caption{Pleiades DANCe DR2 properties}
\begin{center}
\begin{tabular}{|c|c|}

\end{tabular}
\end{center}
\label{tab:DR2Missing}
\end{table}%



\subsection{Selection of observables}
\sloppy
As mentioned earlier the Pleiades DANCe DR2 contains the positions $R.A.$,$Dec.$, proper motions ($\mu_{R.A.},\mu_{Dec.}$ and photometric $ugrizYJHK_s$ bands  of almost 2 million sources on the vicinity of the Pleiades clusters. Although these 13 observables carry information valuable to discriminate cluster members from field objects, not all of them are alike. \citet{Sarro2014} use important analysis with random forest to select the observables that were the most discriminants. Their AM, RF-2, and RF-3 reference sets comprise the $\mu_{R.A.},\mu_{Dec.}$ proper motions and photometric bands $rizYJHK_s$. From them, these authors later excluded the $r$ band because most of the objects in their training set do not have this observable. This resulted in a training set comprising mainly brilliant objects. In the subsequent analysis \citet{Bouy2015} used only the RF-2, which leaves aside the $z$ band.

In this work I only selected $\mu_{\alpha},\mu_{\delta},i,Y,J,H,K_s$ observables since these are the ones used \citet{Bouy2015}. This selection aims to compare our new results with the previous ones \citep{Sarro2014,Bouy2015}. However, for the analysis of the spatial distribution we later added the stellar positions ($R.A., Dec.$).

As will be described later (Section \ref{subsect:cluster}), the photometry is modelled by parametric series of cubic splines. We choose the colour index $i-K_s$ (in the following $CI$) to be the parameter of these series. This colour allows the most one-to-one variate-covariate relation. Figures \ref{fig:CI} and \ref{fig:otherCI} show the two colour-magnitude diagrams $K$ vs $i-K_s$ and $K$ vs $Y-J$. This last one shows second best one-to-one variate-covariate relation. This one-to-one relation is crucial to avoid degeneracies, otherwise two magnitudes could be described by the same colour index and a parametric relation would not be valid. Therefore, our photometric set of observables is $i-K_s, Y,J,H$ and $K_s$. 

\begin{figure}[htbp]
\begin{center}
%\includegraphics[width=\textwidth]{}
\caption{$K_s$ vs $i-K_s$ CMD for the Pleiades candidate members of \citet{Bouy2015}}.
\label{fig:CI}
\end{center}
\end{figure}

\begin{figure}[htbp]
\begin{center}
%\includegraphics[width=\textwidth]{}
\caption{$K_s$ vs $Y-J$ CMD for the Pleiades candidate members of \citet{Bouy2015}}.
\label{fig:otherCI}
\end{center}
\end{figure}

\subsection{Data preprocessing}
Since both photometry and proper motions carry crucial information for the disentanglement of the cluster population, we restrict the data set to only those objects with observed proper motions, and also at least two photometric entries in our photometric set ($i-K_s,Y,J,H,K_s$). These restrictions exclude 22 candidate members of \citet{Bouy2015}. Unfortunately, these objects have only one observed value in the photometry. For these particular objects, we compute their marginalised membership probabilities once the parameters of the model were inferred. Their membership probabilities classify XXXX of them as members. 

 Furthermore, we restrict the lower limit of the $CI$ to the value of the brightest cluster member, $CI =0.8$. We do not expect to find new bluer members in the bright part of the CMDs. Also, we set the upper limit of the CI at one magnitude above the colour index of the reddest known cluster member, $CI=8$, thus allowing for new discoveries. Due to the sensitivity limits of the DR2 survey in $i$ and $K_s$ bands, objects with $CI>8$ have $K_s$ magnitudes $\geq 16$ mag. This combination of CI and $K_s$ magnitude is incompatible with the cluster sequence. Thus, we discard a priori these 695 objects as cluster members. With all the previous restrictions the DANCe DR2 data set was reduced to 1 424 893 objects.

In recent days, our research group has developed a GPU version of the methodology that I present here. This GPU version is about ten times faster than the old CPU one that I developed. However, here I present the results obtained using the CPU version, the fastest one available at that moment.

Our computational constraints (associated to the CPU version) and the costly computations of our methodology (see Sect. \ref{sect:BHM}), prevented its application to the entire data set. However, since the precision of our methodology, as that of any statistical analysis, increases with the number of independent observations, we find that a size of $10^5$ source for our data is a reasonable compromise. Although a smaller data set produces faster results, it also renders a less precise and potentially more biased model of the field (in the area around the cluster) and therefore, a more contaminated model of the cluster. Thus, our data set was restricted to the $10^5$ objects with highest membership probabilities according to \citet{Bouy2015}. The majority of these objects ($\approx$98\%) belong to the field with cluster membership probabilities about zero, according to \citet{Sarro2014,Bouy2015}. Thus, under the assumption that the membership probabilities given by these authors are correct, the probability of leaving out a cluster member is negligible. For the remaining of the objects in the Pleiades DANCe DR2, we assign membership probabilities \emph{a posteriori}, once the cluster model is constructed. 



